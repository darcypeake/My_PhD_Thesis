\chapter{Relevant Algorithms}
\label{app:algorithms}

For completeness, we summarise the relevant DIS clustering algorithms.
\section{The $k_\perp$ algorithm for DIS}
\label{subsec:DIS-kT}
The DIS $k_\perp$ algorithm is a two-step procedure as follows:
\begin{enumerate}
    \item \textbf{Pre-clustering into beam jet and macro-jets}
    \begin{enumerate}
        \item Define a hard-scattering scale, $E_t$, such that $Q^2\geq E_t^2 \gg \Lambda_{\mathrm{QCD}}^2$.
        \vspace{1em}
        
        \item For each particle $i$, compute the beam distance
        \begin{equation}
        	    \label{eq:kT-diB}
            d_{iB} = \frac{2 E_i^2 (1 - \cos\theta_{iB})}{E_t^2}\,,
        \end{equation}
        and for each pair $(i,j)$ compute the distance measure
        \begin{equation}
        	    \label{eq:kT-dij}
            d_{ij} = \frac{2 (1 - \cos\theta_{ij})}{E_t^2} \min(E_i^2, E_j^2)\,.
        \end{equation}
        \vspace{1em}
        
        \item Identify the smallest value among all $\{d_{ij}, d_{iB}\}$:
        \begin{itemize}
            \item If the minimum is of type $d_{ij}$ and  $d_{ij} < 1$, merge $(p_i, p_j)$ into a pseudo-particle $p_{ij}$.
            \vspace{0.5em}
            \item If instead, it is of type $d_{iB}$ and $d_{iB} < 1$, assign $p_i$ to the beam jet.
        \end{itemize}
        \vspace{1em}
        
        \item Repeat the above steps for all particles and pseudo-particles until all
              remaining distances satisfy $d_{ij}, d_{iB} > 1$.
              The result is a beam jet and a set of final-state macro-jets. 
              {\color{red}{Assume the $d_{ij}, d_{iB} > 1$ get assigned to the beam jet.}}
        \end{enumerate}
 \vspace{1em}
    \item \textbf{Resolving the macro-jets substructure}
    \begin{enumerate}
        \item Define a resolution parameter, $d_{\mathrm{cut}}=\frac{Q_0^2}{E_t^2}<1$.
        \item For each particle within the final-state macro-jet, evaluate the distance measures $d_{ij}$ and apply the same clustering procedure as the $k_\perp$ algorithm for $e^+e^-$ annihilation, described in Ref.~\cite{Catani:1991hj}.
    \end{enumerate}
\end{enumerate}
{\color{red}{This algorithm has a different macro-jet meaning to what we have - should perhaps refer to these as pseudo-jets.}}

\section{Cambridge/Aachen algorithms for DIS and Lund variables}
\label{subsec:C/A-DIS-Lund}
The Cambridge/Aachen extension of the DIS algorithm is the following:
\begin{enumerate}
	\item For every final-state parton $i$, define the beam distance
    \begin{equation}
    	\label{eq:Lund-diB}
        d_{iB} = (1 - \cos\theta_{iB})\,,
    \end{equation}
    and for every pair of particles $i$ and $j$, define the angular distance  
    \begin{equation}
    	\label{eq:Lund-dij}
        d_{ij} = (1 - \cos\theta_{ij})\,.
    \end{equation}
   \item Identify the smallest value among all distances in the set $\{d_{ij},\,d_{iB}\}$:
    \begin{itemize}
        \item If the minimum is of type $d_{iB}$, remove parton $i$ from the list and declare it a candidate jet.
        \vspace{0.5em}
        \item If the minimum is of type $d_{ij}$, merge particles $i$ and $j$ into a pseudo-particle using the $E$-scheme ($p^\mu = p_i^\mu + p_j^\mu$) and remove $i$ and $j$ from the list
    \end{itemize}
    \vspace{1em}
    \item Repeat this procedure until only one pseudo-particle remains.  
    This final object is also declared a candidate jet.
\end{enumerate}


\section{Centauro algorithm}
\label{subsec:Centauro}
The Centauro algorithm proceeds analogously to the algorithm described ins~\ref{subsec:C/A-DIS-Lund}, but with a modified distance measure.  
For completeness, we outline its construction.  
Starting from the distance measure in Eq.~\ref{eq:Lund-dij}, one rewrites
\[
1 - \cos\theta_{ij}
    = 1 - \hat{n}_i \!\cdot\! \hat{n}_j
    = 1
      - \sin\theta_i \sin\theta_j \cos\Delta\phi_{ij}
      - \cos\theta_i \cos\theta_j \,.
\]
Expanding in the limit where $\bar{\theta}_i \equiv \pi - \theta_i \ll 1$ yields
\[
1 - \cos\theta_{ij}
   \simeq \frac{1}{2}(\bar{\theta}_i - \bar{\theta}_j)^2
         + \bar{\theta}_i \bar{\theta}_j\, (1 - \cos\Delta\phi_{ij})\,.
\]
Next, the following replacements are made:
\[
\bar{\theta}_i \to f_i \equiv f(\bar{\eta}_i)\,, \qquad
\bar{\eta}_i \equiv -\frac{2Q}{\bar{n}\!\cdot q}\,
                     \frac{p_i^\perp}{n\!\cdot p_i}\,,
\]
with $n^\mu = (1,0,0,+1)$ and $\bar{n}^\mu = (1,0,0,-1)$.  
The function $f(x)$ must satisfy $f(x) = x + \mathcal{O}(x^2)$.
With these definitions, the Centauro distance measure is
\[
d_{ij}
   = \frac{(\Delta f_{ij})^2 + 2 f_i f_j (1 - \cos\Delta\phi_{ij})}{R^2}\,,
\]
and the beam distance is taken to be
\[
d_{iB} = 1\,.
\]


