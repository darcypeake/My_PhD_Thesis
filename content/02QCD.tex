\chapter{Quantum Chromodynamics}
\emph{Include: perturbation theory, UV/IR divergences, subtraction schemes (might need this for explaining the hard function in ARES)}
{\color{green} 
\begin{enumerate}
	\item QCD: quarks, gluons, matrices, Lagrangian, Feynman rules
	\begin{itemize}
	\item The running coupling: asymptotic freedom vs. confinement 
	\end{itemize}
	\item Perturbation theory (refer back to the Feynman rules)
	\begin{itemize}
	\item Divergences: UV and IR
	\item KLN theorem
	\end{itemize}
	\item High-energy physics
	\begin{itemize}
	\item Big picture of HEP + explanation
	\item Factorisation
	\item IRC Safety
	\end{itemize}
	\item Jet Physics
	\begin{itemize}
	\item Jet definition
	\item Types of algorithms
	\item Examples of algorithms
	\end{itemize}	
\end{enumerate}}

\section{QCD: The basics}
{\color{RoyalBlue}
In the 1940s, a number of unstable, strongly interacting particles—collectively known as hadrons—were discovered. In 1964, Gell-Mann~\cite{Gell-Mann:1964ewy} and Zweig~\cite{Zweig:1964ruk, Zweig:1964jf} proposed the quark model, which interpreted hadrons as composite states built from more fundamental spin-$\tfrac{1}{2}$ constituents called \emph{quarks}. The six known quarks are organised into three generations—light, intermediate, and heavy—as listed in Table~\ref{table:quark-properties}. In this picture, hadrons fall into two classes: mesons ($q\bar q$) and baryons ($qqq$).
\begin{table}[h!]
\centering
\begin{tabular}{|c|c|c|c|}
\hline
Light  & Intermediate & Heavy & Charge \\ \hline
$u$ (up)   & $c$ (charm)   & $t$ (top) & $+\tfrac{2}{3}$    \\ \hline
$d$   (down)	& $s$ (strange)    & $b$ (bottom)  & $-\tfrac{1}{3}$  \\ \hline
\end{tabular}
\caption{Properties of the 6 known quarks. More details on the mass can be found in~\cite{ParticleDataGroup:2024cfk}.}
\label{table:quark-properties}
\end{table}

Quantum Chromodynamics (QCD) is an $\mathrm{SU}(3)$ non-Abelian gauge theory that governs the strong interaction. It describes how quarks bind together to form hadrons, with the interaction mediated by the gauge bosons known as gluons. Quarks and gluons are collectively referred to as \emph{partons}, and they carry a conserved charge called colour. Quarks transform in the fundamental representation of $\mathrm{SU}(3)$, with fields denoted by $q_a$ for $a = 1,2,3$, corresponding to the three colour states typically labelled red, green, and blue. The associated antiquarks are written as $\bar{q}^a$. Gluons, denoted $\mathcal{A}^\mu_A$, transform in the adjoint representation of $\mathrm{SU}(3)$, where the index $A$ labels the eight colour degrees of freedom characteristic of the gauge group.\\ \\
The QCD Lagrangian density describes the dynamics of massless gluons and quarks of mass $m$:
\begin{equation}
\label{eq: L-QCD}
\mathcal{L}_{\mathrm{QCD}}
= \mathcal{L}_{\mathrm{classical}}
+ \mathcal{L}_{\mathrm{GF}}
+ \mathcal{L}_{\mathrm{ghost}}\,,
\end{equation}
where the classical part contains the kinetic terms for the gluon and quark fields, together with their interactions:
\begin{equation}
\label{eq: L-Classical}
\mathcal{L}_{\mathrm{classical}}
= -\frac{1}{4} F_{\mu\nu}^{A} F^{\mu\nu}_{A}
+ \sum_{f=1}^{n_f} \bar{q}^f_a (i\slashed{D} - m)_{ab}\, q^f_b \,.
\end{equation}
Here $n_f$ denotes the number of quark flavours, which takes the value $n_f = 6$ in the Standard Model (see Table~\ref{table:quark-properties}). The gluon dynamics are encoded in the non-Abelian field-strength tensor
\begin{equation}
F_{\alpha\beta}^{A}
= \partial_\alpha A_\beta^A
- \partial_\beta A_\alpha^A
- g_s f^{ABC} A_\alpha^B A_\beta^C \,.
\end{equation}
We use the Feynman slash notation $\slashed{D} = \gamma_\mu D^\mu$ for the covariant derivative,
\begin{equation}
(D_\mu)_{ab}
= \partial_\mu \delta_{ab}
+ ig_s (t^C A_\mu^C)_{ab}\,,
\end{equation}
where the gamma matrices satisfy the anti-commutation relation $\{\gamma^\mu,\gamma^\nu\} = 2 g^{\mu\nu}$. Throughout this work we adopt the metric convention $g^{\mu\nu} = (1,-1,-1,-1)$. The matrices $t^C$ denote the generators of the fundamental representation of $\mathrm{SU}(3)$. An explicit representation of these are given by the eight traceless and Hermitian Gell-Mann matrices, $\lambda^C$:
\begin{equation}
	t^C=\frac{1}{2}\lambda^C\,.
\end{equation}
These colour matrices satisfy the commutation relation
\begin{equation}
[t^A, t^B] = i f^{ABC} t^C\,,
\end{equation}
where $f^{ABC}$ are the structure constants of the $\mathrm{SU}(3)$ gauge group. \\ 
In addition to the commutation relation above, the generators obey the identities
\begin{equation}
\begin{aligned}
\mathrm{Tr}(t^A t^B) &= T_R\, \delta^{AB}, \qquad T_R = \frac{1}{2}, \\
\sum_A t^{A}_{ab} t^{A}_{bc} &= C_F\, \delta_{ac}, \qquad C_F = \frac{4}{3}, \\
\sum_{A,B} f^{ABC} f^{ABD} &= C_A\, \delta^{CD}, \qquad C_A = 3.
\end{aligned}
\end{equation}
\\The second term in Eq.~\ref{eq: L-QCD} is the gauge-fixing term:
\begin{equation}
    \mathcal{L}_{\mathrm{GF}}
    = -\frac{1}{2\xi} \left( \partial^\mu A_\mu^{A} \right)^{2}\,.
\end{equation}
This is required since the QCD Lagrangian is invariant under local $\mathrm{SU}(3)$ gauge transformations, which in turn introduces redundant degrees of freedom. This gauge fixing term removes this redundancy. The parameter $\xi$ specifies the choice of gauge; for example, $\xi = 1$ corresponds to Feynman gauge and $\xi = 0$ to Landau gauge.{\color{blue} (Double check that this is correct motivation.)}\\
The third term in Eq.~\ref{eq: L-QCD} is the ghost term,
\begin{equation}
\mathcal{L}_\mathrm{ghost} = \partial_\mu \eta^{A^\dagger}(D_{AB}^\mu\,\eta^B)\,,
\end{equation}
where $\eta^A$ are complex scalar fields known as Faddeev--Popov ghosts. These fields anti-commute and therefore obey Fermi statistics. Ghosts do not correspond to physical states; instead, they appear as a consequence of the gauge-fixing procedure in non-Abelian gauge theories.\\ \\
The Feynman rules extracted from the full QCD Lagrangian are shown in Figs.~\ref{fig:QCD-Propagator-Rules} and~\ref{fig:QCD-Vertex-Rules}. {\color{blue} (Add: the propagators come from the kinetic terms, while the vertices come from the interaction terms and show how various particles couple to one another.)}
For brevity, the ghost Feynman rules are omitted, as they are not required for the calculations presented in this thesis. {\color{red} (Think about whether to say full or not as think it's actually only from the classical part.)}





\begin{figure}[h!]
\centering

\begin{tikzpicture}[baseline=(current bounding box.center)]
\begin{feynman}
    \vertex (a);
    \vertex[right=2cm of a] (b);
    \diagram*{
        (a) -- [fermion, momentum={$p$}] (b),
    };
\end{feynman}
\node[left] at (a) {$a,i$};
\node[right] at (b) {$b,j$};
\node[right=1.0cm of b] {$=\; \delta^{ab}\dfrac{i}{(\slashed{p}-m+i\epsilon)_{ji}}$};
\end{tikzpicture}

\vspace{0.7cm}

\begin{tikzpicture}[baseline=(current bounding box.center)]
\begin{feynman}
    \vertex (a);
    \vertex[right=2cm of a] (b);
    \diagram*{
        (a) -- [gluon, momentum={$p$}] (b),
    };
\end{feynman}
\node[left] at (a) {$A,\mu$};
\node[right] at (b) {$B,\nu$};
\node[right=1.0cm of b] {$=\; \delta^{AB}\dfrac{i}{p^{2}+i\epsilon}\!\left(-g_s^{\mu\nu}-(1-\xi)\dfrac{p^\mu\,p^\nu}{p^2+i\epsilon}\right)$};
\end{tikzpicture}

\caption{Feynman rules for the quark and gluon propagators.}
\label{fig:QCD-Propagator-Rules}
\end{figure}


\begin{figure}[h!]
\centering

\begin{tikzpicture}[baseline=(current bounding box.center)]
\begin{feynman}
    \vertex (a);
    \vertex[right=2cm of a] (b);
    \vertex[above right=1.2cm and 0.8cm of b] (c);
    \vertex[below right=1.2cm and 0.8cm of b] (d);
    \diagram*{
        (a) -- [gluon] (b),
        (b) -- [fermion] (c),
        (b) -- [anti fermion] (d),
    };
\end{feynman}
\node[left] at (a) {$A,\mu$};
\node[right] at (c) {$c,j$};
\node[right] at (d) {$b,i$};
\node[right=1.5cm of b] {$=\; -ig_s\,(t^A)_{cb}\,(\gamma^\alpha)_{ji}$};
\end{tikzpicture}

\vspace{0.7cm}

\begin{tikzpicture}[baseline=(current bounding box.center)]
\begin{feynman}
    \vertex (v);
    \vertex[left=2cm of v] (a);
    \vertex[above right=1.3cm and 0.9cm of v] (b);
    \vertex[below right=1.3cm and 0.9cm of v] (c);

    \diagram*{
        (a) -- [gluon, momentum={$p$}] (v),
        (b) -- [gluon, momentum={$q$}] (v),
        (c) -- [gluon, momentum={$r$}] (v),
    };
\end{feynman}

\node[left] at (a) {$A,\mu$};
\node[above right] at (b) {$B,\nu$};
\node[below right] at (c) {$C,\gamma$};

\node[right=2.3cm of v] {
$\displaystyle
= -g_s\, f^{ABC}
\big[
(p-q)^\gamma g^{\mu\nu}
+ (q-r)^\mu g^{\nu\gamma}
+ (r-p)^\nu g^{\mu\gamma}
\big]
$
};

\end{tikzpicture}


\vspace{0.7cm}
\begin{tikzpicture}[baseline=(current bounding box.center)]
\begin{feynman}
    \vertex (v);
    \vertex[above left=of v]  (A) {$A,\mu$};
    \vertex[below left=of v]  (B) {$B,\nu$};
    \vertex[above right=of v] (C) {$C,\gamma$};
    \vertex[below right=of v] (D) {$D,\delta$};

    \diagram*{
        (v) -- [gluon] (A),
        (v) -- [gluon] (B),
        (v) -- [gluon] (C),
        (v) -- [gluon] (D),
    };
\end{feynman}

\node[right=3.1cm of v] {
$\displaystyle
\begin{aligned}
&= -ig_s^2  \left[f^{EAC}f^{EBD}(g^{\mu\nu}g^{\gamma\delta}-g^{\mu\delta}g^{\nu\gamma})\right.\\
&\qquad\quad + f^{EAD}f^{EBC}(g^{\mu\nu}g^{\gamma\delta}-g^{\mu\gamma}g^{\nu\delta})\\
&\qquad\quad \left.+ f^{EAB}f^{ECD}(g^{\mu\gamma}g^{\nu\delta}-g^{\mu\delta}g^{\nu\gamma})\right]
\end{aligned}
$
};
\end{tikzpicture}


\caption{Feynman rules for the quark--gluon vertex, the three--gluon vertex, and the four--gluon vertex.}
\label{fig:QCD-Vertex-Rules}
\end{figure}
}
This section mainly used the pink book and lecture notes.

\section{The running of the strong coupling $\alpha_s$}
A parameter introduced in the previous section but not yet defined is the coupling $g_s$. This quantity characterises the strength of the strong interaction and is commonly expressed in terms of the strong coupling constant,
\begin{equation}
    \alpha_s \equiv \frac{g_s^2}{4\pi}\,.
    \label{eq:coupling-constant}
\end{equation}
The coupling constant determines the interaction strength and is said to \emph{run} since it varies with the renormalisation scale $\mu_R$ (to be discussed later) according to the renormalisation group equation
\begin{equation}
\label{eq:coupling-DE}
    \mu_R^2 \frac{\partial \alpha_s(\mu_R^2)}{\partial \mu_R^2} = \beta(\alpha_s)\,.
\end{equation}
The $\beta$-function can be expanded perturbatively as
\begin{equation}
    \beta(\alpha_s) = -\!\left(\beta_0\,\alpha_s^2 + \beta_1\,\alpha_s^3 + \beta_2\,\alpha_s^4 + \cdots\right),
\end{equation}
where the coefficients are found to be
\begin{equation}
\begin{aligned}
    \beta_0 &= \frac{11 C_A - 4\,T_R\,n_f}{12\pi}\,, \\
    \beta_1 &= \frac{17 C_A^2 - 5\,C_A\,n_f - 3\,C_F\,n_f}{2\pi\,(11 C_A - 2 n_f)}\,, \\
   & \qquad \qquad\cdots
   % \beta_2 &= \frac{2857 C_A^3 + (54 C_F^2 - 615 C_F C_A - 1415 C_A^2)n_f + (66 C_F + 79 C_A)n_f^2}{288\pi^2\,(11 C_A - 2 n_f)}\,.
\end{aligned}
\end{equation}
he coefficients of the $\beta$-function arise from loop corrections to the bare vertices of the theory. In this way, the scale dependence of the coupling emerges from higher-order quantum corrections. \\\\
%It is worth noting that only $\beta_0$ and $\beta_1$ are renormalisation-scheme independent; $\beta_2$ and higher coefficients depend on the renormalisation scheme (here the $\overline{\mathrm{MS}}$ scheme, discussed later).
 Solving Eq.~\eqref{eq:coupling-DE} to first order, using a reference scale $Q$, yields
\begin{equation}
	\alpha_s(\mu_R^2)=\frac{\alpha_s(Q^2)}{1+\alpha_s\beta_0\ln\tfrac{\mu_R^2}{Q^2}}\,.
\end{equation}
As $\mu_R \to \infty$, the coupling becomes small, allowing quarks and gluons to behave as nearly free particles. This phenomenon is known as \emph{asymptotic freedom}, a regime in which perturbation theory (see below) provides us with a very good approximation. Conversely, as $\mu_R \to 0$, the coupling becomes large, signalling the \emph{confinement} regime in which quarks and gluons are bound into hadrons and perturbation theory breaks down. \\\\
The effect of the running coupling has also been confirmed experimentally, as illustrated in Fig.~\ref{fig:strong-coupling}.
\begin{figure}[h]
    \centering
    \includegraphics[width=\linewidth]{figures/strong_coupling.pdf}
    \caption{Summary of determinations of $\alpha_s$ as a function of the energy scale $Q$, compared to the running of the coupling computed at five loops using the current PDG average $\alpha_s(m_Z^2) = 0.1180 \pm 0.0009$~\cite{ParticleDataGroup:2024cfk}. {\color{blue}(Need to change this.)}}
    \label{fig:strong-coupling}
\end{figure}
Throughout this thesis, we will assume we are working in the high-energy limit, exploiting asymptotic freedom.
This section used the pink and black book.

\section{Perturbation Theory}

With the QCD Lagrangian and Feynman rules established in the previous section, we can now outline how calculations are carried out in practice. At high energies, the strong coupling $\alpha_s$ becomes small, which allows the interaction terms in the Lagrangian to be treated as a perturbation of the free theory. Using the propagators and vertices shown in Figs.~\ref{fig:QCD-Propagator-Rules} and~\ref{fig:QCD-Vertex-Rules}, we construct Feynman diagrams whose contributions are organised according to their powers of~$\alpha_s$.

The resulting scattering amplitude (matrix element) takes the form of a perturbative series,
\begin{equation}
    \mathcal{M}
    = \mathcal{M}_0
    + \alpha_s\,\mathcal{M}_1
    + \alpha_s^2\,\mathcal{M}_2
    + \cdots\,,
    \label{eq:pQCD}
\end{equation}
where $\mathcal{M}_0$ is the tree-level term and $\mathcal{M}_i$ denotes the $i$-th order correction. Since $\alpha_s$ is small in the high-energy regime, higher-order terms are increasingly suppressed, meaning that truncating the series after the first few contributions typically yields a reliable approximation. Such a truncated calculation is referred to as a fixed-order (FO) prediction. The first non-zero term is called leading order (LO), the next term next-to-leading order (NLO), followed by next-to-next-to-leading order (NNLO), and so on.

The introduction of loop diagrams at NLO and beyond brings new features—most notably ultraviolet and infrared divergences—which require regularisation and renormalisation. These issues will be discussed in the next section.

CHANGE ALL OF THIS TO OWN WORDS \\ \\
SWAP WITH PREVIOUS SECTION \\ \\
SEE IF WE NEED TO REFERENCE \\ \\ 
SAY IN HIGH ENERGY LIMIT, SMALL COUPLING.

\section{Divergences}
\subsection{UV Divergences}
\subsection{IR Divergences}
{\color{blue} Need to reword this whole section - start back from here.} \\ \\ 
While perturbation theory is a fantastic tool used in the high-energy limit - it does come with problems in the form of divergences{\color{blue} (are these divergences problems?)}. There are two types of divergences, the first referred to as ultraviolet (UV) divergences. This occurs when considering a loop diagram such that we integrate over the momentum of a virtual particle {\color{blue} (maybe a little more information on this and fact check)}. The other type that occur are known as infrared and collinear (IRC) divergences {\color{blue} (can we just call them IR divergences?)}. {\color{blue} (Better way to write this).} In order to investigate these IRC divergences, let us consider the simplest QCD process: the production of a quark-antiquark pair from either a virtual photon or a $Z$-boson ($e^+e^-\rightarrow q\bar{q}$) \cite{Luisoni:2015xha}, \cite{McAslan:2017bqp}, \cite{Arpino:2020smn}. In this example, we choose to look at the case of the photon where the LO diagram is shown in figure \ref{fig:photon-qq}. 
\begin{figure}[h]
    \centering
    \begin{tikzpicture}
        \begin{feynman}
            \vertex (a) {\(\gamma\)};
            \vertex [right=of a] (b);
            \vertex [above right=of b] (f1) {\(q\)};
            \vertex [below right=of b] (f2) {\(\overline{q}\)};
            
            \diagram* {
                (a) -- [photon] (b) -- [fermion] (f1),
                           (b) -- [anti fermion] (f2),
            };
        \end{feynman}
    \end{tikzpicture}
    \caption{Feynman diagram showing the production of $q\bar{q}$ from a photon (LO process).}
    \label{fig:photon-qq}
\end{figure}
The NLO diagram process can be shown in figures \ref{fig:photon-qg} and \ref{fig:photon-bar-qg} and depicts the process where an additional gluon is radiated from either  $q$ or $\bar{q}$ ($e^+e^-\rightarrow q\bar{q}g$). 
{\color{green} Add figures.}


We wish to consider the NLO cross-section, which we will not calculate but simply state the result:
 \begin{equation}
	\sigma_{\gamma\rightarrow q\bar{q}g}=\int[d\Phi_{3}]\,|M_{\gamma\rightarrow q\bar{q}g}|^2=C_F\,g_s^2\int[d\Phi_{2}]\,|M_{q\bar{q}}|^2\,\frac{d^3\vec{k}}{(2\pi)^32E_g}\Theta(E_g)\frac{2p_1p_2}{(p_1k)(p_2k)}\,,
	\label{eq:qqg-cross-section}
\end{equation}
where $E_q$, $E_{\bar{q}},$ and $E_g$ are the energies of the quark, anti-quark and gluon respectively. In the first step, we integrate with respect to the three-body phase space $d\Phi_{3}$, since there are three-particles present in the final state. The second equality holds only if we assume $E_g\ll E_q,E_{\bar{q}}$ \emph{18/09/24 i.e. in the limit where the gluon is soft such that it factorises into a LO matrix element squared and a real gluon emission}. In this case we see that the total cross-section can be factorised into a $q\bar{q}$ piece and a gluon piece. The $q\bar{q}$ piece is the cross-section at leading order, $\int[d\Phi_{2}]\,|M_{q\bar{q}}|^2$, whereas the gluon piece is the following:%Check this is all true - is this the idea that it can only happen when the gluon is soft? Is it worthwhile mentioning the hard process of e+e-->qbar{q}?
\begin{equation}
	\sigma_g=C_F\,g_s^2\int\frac{d^3\vec{k}}{2E_g((2\pi)^3}\Theta(E_g)\frac{2p_1p_2}{(p_1k)(p_2k)}=\frac{2\alpha_sC_F}{\pi}\int\frac{dE_g}{E_g}\frac{d\phi}{2\pi}\frac{d\theta}{\sin\theta}\,.
	\label{eq:g-cross-section}
\end{equation}
The angle $\theta$ is between the gluon and the quark (or anti-quark, depending on which is emitting the gluon) and $\phi$ is the angle in the out-of-page direction. \emph{18/09/2024 Note that equation~\eqref{eq:qqg-cross-section} is essentially like saying that the probability of an emission of a soft gluon from a hard emitter is given by the product of the LO $q\bar{q}$ production probability and the soft emission probability shown in equation~\eqref{eq:g-cross-section}}. The soft emission probability shows that there are singularities as the gluon becomes soft ($E_g\rightarrow0$) and when the gluon becomes collinear to the emitting quark (or anti-quark) ($\theta\rightarrow0,\pi$).
These are the IRC divergences talked about previously, and will arise for every gluon emission and therefore at every order in perturbation theory. When fully integrated (using dimensional regularisation to regulate the integrals), we obtain values proportional to $\frac{1}{\epsilon^2}$,  coming from the region where the gluon is soft and collinear, and terms proportional to $\frac{1}{\epsilon}$, which occur due to the gluon being only collinear to the quark (or antiquark). 
\begin{figure}[h]
    \centering
    \begin{tikzpicture}
        \begin{feynman}
            % Define vertices
            \vertex (a) {\(\gamma\)};
            \vertex [right=of a] (b);
            \vertex [above right=of b] (f1) {\(q\)};
            \vertex [below right=of b] (f2) {\(\overline{q}\)};
            \vertex at ($(f1)!0.5!(b)$) (mid1);
            \vertex at ($(f2)!0.5!(b)$) (mid2);
            
            % Draw diagram
            \diagram* {
                (a) -- [photon] (b),
                (b) -- [fermion] (f1),
                (b) -- [anti fermion] (f2),
                (mid1) -- [gluon] (mid2),
            };
        \end{feynman}
    \end{tikzpicture}
    \caption{NLO virtual diagram for $\gamma\rightarrow q\bar{q}g$.}
    \label{fig:photon-qq-gluon}
\end{figure}
The diagrams in figures \ref{fig:photon-qg} and \ref{fig:photon-bar-qg}, show the emission of a real gluon and therefore are termed real diagrams. In pQCD, it mandatory for us to include all diagrams which are allowed at a given order, and so we must also consider the diagram where there is a virtual gluon emitted and received between the quark-antiquark pair as depicted in figure \ref{fig:photon-qq-gluon}. We find that the cross-section of this process also contains divergences, and are equal and opposite to those which come from the real NLO processes. Therefore, we find a finite result when computing the overall cross-section at NLO \cite{Schwartz:2014sze}, \cite{Luisoni:2015xha}. We will state again for emphasis that this only occurs when the gluon is soft ($E_g\ll E_q,E_{\bar{q}}$). Next, we will see whether this behaviour occurs for other observables, beyond the cross-section. \emph{18/09/2024 It is important to note that it is the KLN theorem which states that the cancellation of IRC divergences between real and virtual diagrams occur at every order in perturbation theory provided that we sum over all possible virtual and final states.}



\section{High-energy physics}
\subsection{Explanation of Process}
\subsection{Factorisation}
\subsection{Parton Model (maybe in previous section)}
Give explanation at the beginning. 
\emph{Include: factorisation at high-energy limit, what a jet is, the process following a high-energy collision, IRC Safety, resummation requirement.}
{\color{green} Add in the big picture thing that is used in every single thesis here I think.}
\subsection{IRC Safety} 
\label{sec: IRC}
The cross-section falls into a category of inclusive observables. For such observables, the soft and collinear divergences cancel completely between the real and virtual diagrams in the IR limit (where $E_g\ll E_q,E_{\bar{q}}$) \cite{Campbell:2017hsr}. Exclusive observables, on the other hand, contain singularities and therefore unphysical results. Therefore, in order to avoid the issue of infinities we must define these to be IRC safe observables, which ensure the divergences cancel \cite{Luisoni:2015xha}, \cite{McAslan:2017bqp}, \cite{Arpino:2020smn}. \\ \\
An IRC safe observable is one that does not change regardless of the number of soft or collinear emissions, such that it obeys the following:
\begin{equation}
V(\{q\},k_1,\cdots ,k_n)=V(\{q\},k_1,\cdots , k_{j+1},\cdots ,k_n),\quad \text{when}\quad E_{j+1}\rightarrow0\quad\text{or}\quad\vec{k}_{j+1}\parallel \vec{q}_{\text{emit}}
	\label{eq:IRC-safety}
\end{equation}
where $V$ is the observable in question, $\{q\}$ are the hard partons, $k_{j+1}$ is the additional emission and ${q}_{\text{emit}}$ is the parent parton which emits the additional gluon. While equation~\eqref{eq:IRC-safety} holds for a single emission, the above must be true for any number of emissions. Therefore, IRC safe observables can be calculated to all orders in perturbation theory.




\section{Jet Physics}
{\color{RoyalBlue}
Jets are among the most frequently observed objects in proton–proton collisions at the LHC. They appear as highly collimated sprays of hadrons resulting from the hadronisation of energetic quarks and gluons produced in high-energy processes~\cite{Banfi:2016yyq}. When a hard parton is created in a short-distance interaction, it radiates gluons in a similarly collimated pattern along its initial direction. These gluons can in turn branch into further gluons or quark–antiquark pairs~\cite{Schwartz:2014sze}. As the radiation proceeds, the typical energy scale of the process decreases and the interactions between the partons become stronger, until they reach an energy of order $\Lambda_{\mathrm{QCD}}$. At this stage, the partons cluster together to form the final-state hadrons observed in the detector. These hadrons tend to group into fairly energetic, collimated clumps - known as \emph{jets}~\cite{Schwartz:2014sze, Campbell:2017hsr, Banfi:2016yyq}.
}




\subsection{Jet Definition}
{\color{RoyalBlue}
As described above, jets are complex, messy, and inherently ambiguous objects. To be able to describe them in a well-defined and reproducible way, we must introduce a \emph{jet definition}, which consists of the following components~\cite{Salam:2010nqg}:
\begin{enumerate}
\item a \emph{jet algorithm}: a systematic procedure for grouping final-state particles into jets and determining the number of jets present in an event, i.e.\ a set of rules that cluster particles into a given jet;
\item the algorithm's \emph{parameters}: these specify the proximity required between two particles for them to be recombined into a single entity belonging to the same jet;
\item a \emph{recombination scheme}: this defines how the momenta of two particles are combined to form the momentum of the new clustered particle — the most commonly used being the $E$-scheme, , which is a four-momentum sum, i.e.\ $p_i^\mu + p_j^\mu = p_k^\mu$.
\end{enumerate}
In the next subsection, we discuss the jet algorithm and its input parameters in more detail. }

\subsection{Jet Algorithm}
{\color{RoyalBlue}
In 1990, there was significant discussion within the high-energy physics community about how jets should be properly defined. To address this, a group of physicists formulated a set of requirements that a well-defined jet algorithm should satisfy, outlined in a document known as the \emph{Snowmass Accord}~\cite{Huth:1990mi}. The criteria are as follows:
\begin{enumerate}
    \item Simple to implement in an experimental analysis;
    \item Simple to implement in a theoretical calculation;
    \item Defined at any order in perturbation theory;
    \item Yields a finite cross section at any order in perturbation theory;
    \item Yields a cross section that is relatively insensitive to hadronisation effects.
\end{enumerate}
IRC safety ensures that jets defined at the detector, hadron, and parton levels are essentially equivalent, thereby satisfying the final three requirements outlined above. For IRC-safe observables, the effects of hadronisation are suppressed by inverse powers of the energy scale used in the process. Consequently, at higher energies, the correspondence between hadronic and partonic observables becomes increasingly accurate~\cite{Banfi:2016yyq}.
Overall, these requirements allow us to to regard jets as well-defined objects whose properties can be either determined by their constituent hadrons or underlying partons. This enables direct comparison between experimental measurements and theoretical calculations at the parton level. \\ \\
A wide variety of jet algorithms exist, each with its own advantages and limitations. The choice of algorithm depends on the specific physics question being addressed and the type of information one seeks to extract from an event. In general, jet algorithms can be divided into two broad classes: \emph{sequential recombination algorithms} and \emph{cone algorithms}~\cite{Banfi:2016yyq},~\cite{Salam:2010nqg}.
\begin{itemize}
    \item \textbf{Cone algorithms:} These group particles within a fixed radius $R$ in $(\eta,\phi)$ space, such that the particles whose transverse energy deposits fall inside the circular region are clustered into a jet. In three dimensions, these regions appear as cones. Different implementations vary in how they search for and define these cones. Cone algorithms are often described as \emph{top-down} algorithms.

    \item \textbf{Sequential recombination algorithms:} These iteratively cluster the closest particles according to a chosen distance measure until a stopping criterion is reached. Variations arise from different definitions of the distance measure (e.g.\ relative transverse momentum or angular separation between particles) and from the choice of stopping condition. Sequential recombination algorithms are often referred to as \emph{bottom-up} algorithms.
\end{itemize}
In this thesis, we focus exclusively on sequential recombination algorithms. The following sections discuss several of these algorithms to provide background for the new DIS algorithm introduced in Chapter~\ref{chapter: DIS}. For a comprehensive review of jets and jet algorithms across different processes, see Ref.~\cite{Salam:2010nqg}.
}

\subsection{Jet Algorithms for $e^+e^-$ Collisions}
\label{subsec: e+e-}
{\color{RoyalBlue}
Jet algorithms for electron–positron collisions follow a common iterative clustering procedure. Each pair of particles is assigned a \emph{distance measure}, $d_{ij}$, which determines how closely they are related. The closest pair is iteratively recombined until all remaining objects are separated by more than the chosen \emph{resolution parameter}, $d_{\mathrm{cut}}$.
This framework forms the basis of the JADE and Durham algorithms, while the Cambridge algorithm extends it by introducing an additional \emph{ordering variable}, $v_{ij}$, as discussed in the following sections.

\subsubsection{The JADE algorithm}
The first sequential recombination algorithm was introduced by the JADE Collaboration in 1988~\cite{JADE:1988xlj} and is defined as follows:
\begin{enumerate}
    \item \textbf{Define a resolution parameter} $d_{\mathrm{cut}}$.
    \item \textbf{Compute the distance measure} for every pair of particles $i$ and $j$:
    \begin{equation}
        \label{eq:yij-jade}
        d^{\mathrm{JADE}}_{ij} = \frac{2E_iE_j(1 - \cos\theta_{ij})}{Q^2}\,,
    \end{equation}
    where $Q$ is the total centre-of-mass energy, $E_i$ is the energy of particle $i$, and $\theta_{ij}$ is the angle between particles $i$ and $j$.
    \item \textbf{Identify the smallest $d_{ij}$ value.} If this minimum satisfies $d_{ij} < d_{\mathrm{cut}}$, the two particles are recombined into a pseudo-particle according to the chosen recombination scheme.
    \item \textbf{Repeat the procedure} from step 2 until all remaining pairs satisfy $d_{ij} > d_{\mathrm{cut}}$. The remaining objects are then defined as jets.
\end{enumerate}
A drawback of the JADE algorithm is its tendency to cluster soft pairs of gluons that are widely separated in angle. This can lead to situations where a hard parton is incorrectly merged with a soft gluon that was not emitted from it. Physically speaking, a jet should constitute to a hard parton together with its associated radiation, so such clustering is undesirable. This issue is resolved by modifying the distance measure, leading to the $k_\perp$ algorithm.
}


\subsubsection{ The $k_\perp$ (Durham) algorithm}
{\color{RoyalBlue}
Also known as the Durham algorithm~\cite{Catani:1991hj}, this approach follows the same procedure as JADE but employs a modified distance measure:
\begin{equation}
	 \label{eq:yij-De}
    	d^{\mathrm{Durham}}_{ij}=\frac{2(1-\cos\theta_{ij})}{Q^2}\min(E_i^2,E_j^2)\,.
\end{equation} 
At small angles, the numerator can be approximated to $(\min(E_i,E_j)\theta_{ij})^2$, which corresponds to the squared relative transverse momentum of particle $i$ with respect to particle $j$ (for $E_i < E_j$). The use of the $\min$ function ensures that soft emissions, widely separated in angle, have a larger distance measure than those corresponding to a hard parton radiating a nearby soft gluon. This modification prevents unphysical clustering of unrelated soft particles and produces jets that more accurately reflect the underlying partonic structure.
}
\subsubsection{The Cambridge algorithm}
{\color{RoyalBlue}
The Cambridge algorithm, first introduced in Ref.~\cite{Dokshitzer:1997in}, extends the Durham algorithm by employing angular ordering rather than transverse-momentum ordering. Unlike the previous two algorithms, it introduces an additional \emph{ordering variable}, $v_{ij}$, used alongside the distance measure $d^{\mathrm{Durham}}_{ij}$ from the Durham algorithm:
\begin{enumerate}
    \item \textbf{If only one particle remains,} stop the clustering and define this object as a jet.
    \item \textbf{Find the smallest $v_{ij}$} among all pairs of particles:
    \begin{equation}
        \label{eq:vij-cambridge}
        v_{ij} = 2(1 - \cos\theta_{ij})\,.
    \end{equation}
    \item \textbf{Identify the corresponding $d_{ij}$ value.} If $d_{ij} < d_{\mathrm{cut}}$, recombine the particles as in the Durham algorithm and return to step 1.
    \item \textbf{Otherwise,} remove the less energetic particle, label it as a jet, and return to step 1.
\end{enumerate}
This clustering procedure effectively reconstructs the sequence of gluon emissions in reverse, which typically occur at progressively smaller angles~\cite{Banfi:2016yyq}. As a result, the Cambridge algorithm is a better algorithm for the resolution of jet substructure and for reducing non- perturbative effects, which occur since emissions widely separated in angles are emitted independently from the hard legs. 
}
{\color{red} (Not understanding why this will give rise to better substructure and reduced NP effects.) }

\subsection{Jet Algorithms for Deep-Inelastic Scattering}
\label{subsec: DIS-alg}
{\color{RoyalBlue}
 Deep-inelastic scattering (DIS) is the process:
\begin{equation}
\label{eq: DIS}
    e^-(k) + N(P) \rightarrow e^-(k') + X(p_X)\,,
\end{equation}
where $N$ denotes the incoming hadron (usually a proton) and $X$ represents all final-state hadrons produced. In DIS processes, three lorentz invariants are often used to describe the kinematics:
\begin{equation}
\begin{aligned}
Q^2=-q^2\,, \qquad x=\frac{Q^2}{2P\cdot q}\,, \qquad y=\frac{P\cdot q}{P\cdot k}\,,
\end{aligned}
\end{equation}
where $Q^2$ is the virtuality of the photon, $x$ is the Bjorken $x$ variable (fraction of momentum taken by the struck quark from the the incoming hadron), and $y$ is the energy transferred between leptonic and hadronic systems~\cite{Devenish:2004pb}.  {\color{blue}(Need to get some more information on invariants (particularly $y$.)}
 \\ \\
At LO, the process produces a single outgoing parton in its final hadronic state:
\begin{equation}
    e^-(k) + N(P) \rightarrow e^-(k') + p_{\mathrm{out}}(p')\,.
\end{equation}
DIS processes are often analysed in the \emph{Breit frame}, defined as the reference frame in which the exchanged virtual photon carries no energy component, with four-momentum $q^{\mu} = (0, 0, 0, -Q)$. In this frame $P^{\mu} = p^{\mu}/x$, where  $ p^{\mu}$ is the momentum of the incoming parton, $p_{\mathrm{in}}$. In the Breit frame, the momenta of the incoming and outgoing partons are:  
\begin{equation}
\begin{aligned}
    p^{\mu} &= \frac{Q}{2}(1, 0, 0, +1)\,, \qquad
    p'^{\mu} = \frac{Q}{2}(1, 0, 0, -1)\,.
\end{aligned}
\end{equation}
In this frame, the proton and the virtual photon collide head-on, providing a clear separation between the incoming proton and the struck quark.\\ \\
 The presence of an initial-state hadron requires a modification of the $e^+e^-$ jet algorithms described previously to account for the incoming beam. In sequential recombination algorithms, this can be achieved by introducing an additional quantity, the \emph{beam distance}, $d_{iB}$~\cite{Salam:2010nqg}
\footnote{In hadron–hadron collisions, two beam distances are defined, $d_{iB}$ and $d_{i\bar{B}}$, corresponding to the two incoming beams.}.} \\ \\
{\color{blue} (I think we need some information on the variables i.e. $x$, $y$, and $Q^2$, along with a few images of the lab frame DIS and the Breit frame DIS, and also some information on TFR and CFR.)}


\subsubsection{$k_\perp$ DIS Algorithm}
{\color{RoyalBlue}
As introduced in Ref.~\cite{Catani:1992zp}, the $k_\perp$ algorithm described in Section~\ref{subsec: e+e-} was extended to DIS. The procedure is defined in the Breit frame and is performed in two stages. The aim is to first cluster all particles into a \emph{beam jet} and  \emph{final-state macro-jets}, and then to resolve the jet sub-structure within the latter.
\begin{enumerate}
    \item \textbf{Pre-clustering into beam jet and macro-jets}
    \begin{enumerate}
        \item Define a hard-scattering scale, $E_t$, such that $E_t^2 \gg \Lambda_{\mathrm{QCD}}^2$.
        \item For each particle $i$, compute the beam distance:
        \begin{equation}
        \label{eq:DIS-durham-diB}
            d_{iB} = \frac{2E_i^2(1 - \cos\theta_{iB})}{E_t^2}\,,
        \end{equation}
        where $\theta_{iB}$ is the angle between particle $i$ and the beam (initial-state proton) direction.

        \item Compute the distance measure for every pair of particles $i$ and $j$:
        \begin{equation}
        \label{eq:DIS-durham-dij}
            d_{ij} = \frac{2(1 - \cos\theta_{ij})}{E_t^2}\min(E_i^2, E_j^2)\,.
        \end{equation}

        \item Identify the smallest value among $\{d_{ij}, d_{iB}\}$:
        \begin{itemize}
            \item If $d_{ij} < 1$ and is the smallest, combine $(p_i, p_j)$ into a pseudo-particle $p_{ij}$, using the $E$-scheme recombination scheme.
            \item If $d_{iB} < 1$ and is the smallest, assign $p_i$ to the beam jet.
        \end{itemize}
        \item Repeat the procedure iteratively from step~(b) for all particles and pseudo-particles not yet assigned to the beam jet, until all remaining objects satisfy $d_{ij}, d_{iB} > 1$. The result is a final set of clustered objects consisting of a beam jet and final-state macro-jets. {\color{red}(I am unsure what happens to the particles which are greater than 1. I think they are just macrojets. I am also unsure if there is 1 or multiple macrojets)}
    \end{enumerate}

    \item \textbf{Resolving the jet structure of the macro-jet}
    \begin{enumerate}
        \item Define a resolution parameter, $d_{\mathrm{cut}}$.
        \item For each particle within the final-state macro-jet, evaluate the distance measures $d_{ij}$ and apply the same clustering procedure as the $k_\perp$ algorithm for $e^+e^-$ annihilation, described in Section~\ref{subsec: e+e-} to the final-state macrojet.
    \end{enumerate}
\end{enumerate}
}


\subsubsection{Cambridge DIS Algorithm}
{\color{RoyalBlue}
The {\color{blue}(exclusive)} Cambridge algorithm for deep-inelastic scattering, originally proposed in Ref.~\cite{Wobisch:1998wt}, is an extension of the $e^+e^-$ Cambridge algorithm introduced in Section~\ref{subsec: e+e-}, generalised in the same manner as the $k_\perp$ algorithm was extended from $e^+e^-$ collisions to DIS, in the section above. \\\\
The procedure follows the same steps as the $k_\perp$ DIS algorithm described above, but uses angular \emph{ordering variables}, $v_{ij}$ and $v_{iB}$, in place of the distance measures $d_{ij}$ and $d_{iB}$. The variable $v_{ij}$ is defined in Eq.~\ref{eq:vij-cambridge}, while the beam angular-ordering variable $v_{iB}$ is given by:
\begin{equation}
    v_{iB} = 2(1 - \cos\theta_{iB})\,.
\end{equation}
The clustering proceeds by finding the smallest value among $\{v_{ij}, v_{iB}\}$ and determining whether the corresponding object is associated with a final-state macro-jet or the beam jet. Once the pre-clustering is complete, the algorithm proceeds as in the $e^+e^-$ Cambridge algorithm (see Section~\ref{subsec: e+e-}) to resolve the jet substructure within each macro-jet.
}
{\color{blue} (Add in: widely separated in angle are attached to the hard leg i.e. no multiple gluon clustering - soft jet freezing.)}







