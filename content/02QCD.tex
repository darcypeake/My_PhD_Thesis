\chapter{Quantum Chromodynamics}
\section{QCD: The basics}
In the 1940s, a number of unstable, strongly interacting particles -- collectively referred to as hadrons -- were discovered. In 1964, Gell-Mann~\cite{Gell-Mann:1964ewy} and Zweig~\cite{Zweig:1964ruk, Zweig:1964jf} proposed the quark model, which interprets hadrons as composite objects built from fundamental spin-$\tfrac{1}{2}$ constituents known as \emph{quarks}. The six known quarks are organised into three generations -- light, intermediate, and heavy -- as listed in Table~\ref{table:quark-properties}. In this picture, hadrons fall into two classes: mesons ($q\bar q$) and baryons ($qqq$).
\begin{table}[h!]
\centering
\begin{tabular}{|c|c|c|c|}
\hline
Light  & Intermediate & Heavy & Charge \\ \hline
$u$ (up)   & $c$ (charm)   & $t$ (top) & $+\tfrac{2}{3}$    \\ \hline
$d$   (down)	& $s$ (strange)    & $b$ (bottom)  & $-\tfrac{1}{3}$  \\ \hline
\end{tabular}
\caption{Properties of the 6 known quarks. More details on the mass can be found in~\cite{ParticleDataGroup:2024cfk}.}
\label{table:quark-properties}
\end{table}

Quantum Chromodynamics (QCD) is an $\mathrm{SU}(3)$ non-Abelian gauge theory that governs the strong interaction. It describes how quarks are bound together to form hadrons, with the interaction mediated by gauge bosons known as gluons. Quarks and gluons are collectively referred to as \emph{partons} and carry a conserved charge called colour. Quarks transform in the fundamental representation of $\mathrm{SU}(3)$, with fields denoted by $q_a$ for $a = 1,2,3$, corresponding to the three colour labels: red, green, and blue. The associated antiquark fields are written as $\bar q^a$. Gluons, denoted by $\mathcal{A}^\mu_A$, transform in the adjoint representation of $\mathrm{SU}(3)$, where the index $A$ labels the eight colour degrees of freedom of the gauge group.\\ \\
The QCD Lagrangian density describes the dynamics of massless gluons and quarks of mass $m$:
\begin{equation}
\label{eq: L-QCD}
\mathcal{L}_{\mathrm{QCD}}
= \mathcal{L}_{\mathrm{classical}}
+ \mathcal{L}_{\mathrm{GF}}
+ \mathcal{L}_{\mathrm{ghost}}\,,
\end{equation}
where the classical contribution contains the kinetic and interaction terms for the quark and gluon fields:
\begin{equation}
\label{eq: L-Classical}
\mathcal{L}_{\mathrm{classical}}
= -\frac{1}{4} F_{\mu\nu}^{A} F^{\mu\nu}_{A}
+ \sum_{f=1}^{n_f} \bar{q}^f_a (i\slashed{D} - m)_{ab}\, q^f_b \,.
\end{equation}
Here $n_f$ denotes the number of quark flavours, which takes the value $n_f = 6$ in the Standard Model (see Table~\ref{table:quark-properties}). The gluon dynamics are encoded in the non-Abelian field-strength tensor
\begin{equation}
F_{\alpha\beta}^{A}
= \partial_\alpha A_\beta^A
- \partial_\beta A_\alpha^A
- g_s f^{ABC} A_\alpha^B A_\beta^C \,,
\end{equation}
where the strength of the interaction is governed by the strong coupling $g_s$ which can be expressed in terms of the coupling constant, 
\begin{equation}
    \alpha_s \equiv \frac{g_s^2}{4\pi}\,.
    \label{eq:coupling-constant}
\end{equation}
The covariant derivative is written using Feynman slash notation, $\slashed{D} = \gamma_\mu D^\mu$, with
\begin{equation}
(D_\mu)_{ab}
= \partial_\mu \delta_{ab}
+ ig_s (t^C A_\mu^C)_{ab}\,,
\end{equation}
where the gamma matrices satisfy the anti-commutation relation $\{\gamma^\mu,\gamma^\nu\} = 2 g^{\mu\nu}$. Throughout this work we adopt the metric convention $g^{\mu\nu} = (1,-1,-1,-1)$. The matrices $t^C$ denote the generators of the fundamental representation of $\mathrm{SU}(3)$. An explicit representation is given by the eight traceless and Hermitian Gell-Mann matrices, $\lambda^C$, such that
\begin{equation}
	t^C=\frac{1}{2}\lambda^C\,.
\end{equation}
These colour matrices satisfy the commutation relation
\begin{equation}
[t^A, t^B] = i f^{ABC} t^C\,,
\end{equation}
where $f^{ABC}$ are the structure constants of the $\mathrm{SU}(3)$ gauge group. In addition to the commutation relation above, the generators obey the identities
\begin{equation}
\begin{aligned}
\mathrm{Tr}(t^A t^B) &= T_R\, \delta^{AB}, \qquad T_R = \frac{1}{2}, \\
\sum_A t^{A}_{ab} t^{A}_{bc} &= C_F\, \delta_{ac}, \qquad C_F = \frac{4}{3}, \\
\sum_{A,B} f^{ABC} f^{ABD} &= C_A\, \delta^{CD}, \qquad C_A = 3.
\end{aligned}
\end{equation}
\\The second term in Eq.~\ref{eq: L-QCD} is the gauge-fixing term:
\begin{equation}
    \mathcal{L}_{\mathrm{GF}}
    = -\frac{1}{2\xi} \left( \partial^\mu A_\mu^{A} \right)^{2}\,.
\end{equation}
This term is required due to the local $\mathrm{SU}(3)$ gauge invariance of the QCD Lagrangian, which leads to redundant gauge degrees of freedom. The gauge-fixing term removes this redundancy, and the parameter $\xi$ specifies the choice of gauge; for example, $\xi = 1$ corresponds to Feynman gauge, while $\xi = 0$ corresponds to Landau gauge.
\\\\
The final term in Eq.~\ref{eq: L-QCD} is the ghost term,
\begin{equation}
\mathcal{L}_\mathrm{ghost} = \partial_\mu \eta^{A^\dagger}(D_{AB}^\mu\,\eta^B)\,,
\end{equation}
where $\eta^A$ are complex scalar fields known as Faddeev--Popov ghosts. These fields anti-commute and therefore obey Fermi statistics. Ghosts do not correspond to physical states; instead, they arise as a consequence of the gauge-fixing procedure in non-Abelian gauge theories.\\ \\
The Feynman rules derived from the QCD Lagrangian are shown in Figs.~\ref{fig:QCD-Propagator-Rules} and~\ref{fig:QCD-Vertex-Rules}. The propagators follow from the kinetic terms of the Lagrangian, while the interaction terms give rise to the vertices. For brevity, the ghost Feynman rules are omitted, as they are not required for the calculations presented in this thesis.

\begin{figure}[h!]
\centering

\begin{tikzpicture}[baseline=(current bounding box.center)]
\begin{feynman}
    \vertex (a);
    \vertex[right=2cm of a] (b);
    \diagram*{
        (a) -- [fermion, momentum={$p$}] (b),
    };
\end{feynman}
\node[left] at (a) {$a,i$};
\node[right] at (b) {$b,j$};
\node[right=1.0cm of b] {$=\; \delta^{ab}\dfrac{i}{(\slashed{p}-m+i\epsilon)_{ji}}$};
\end{tikzpicture}

\vspace{0.7cm}

\begin{tikzpicture}[baseline=(current bounding box.center)]
\begin{feynman}
    \vertex (a);
    \vertex[right=2cm of a] (b);
    \diagram*{
        (a) -- [gluon, momentum={$p$}] (b),
    };
\end{feynman}
\node[left] at (a) {$A,\mu$};
\node[right] at (b) {$B,\nu$};
\node[right=1.0cm of b] {$=\; \delta^{AB}\dfrac{i}{p^{2}+i\epsilon}\!\left(-g_s^{\mu\nu}-(1-\xi)\dfrac{p^\mu\,p^\nu}{p^2+i\epsilon}\right)$};
\end{tikzpicture}

\caption{Feynman rules for the quark and gluon propagators.}
\label{fig:QCD-Propagator-Rules}
\end{figure}


\begin{figure}[h!]
\centering

\begin{tikzpicture}[baseline=(current bounding box.center)]
\begin{feynman}
    \vertex (a);
    \vertex[right=2cm of a] (b);
    \vertex[above right=1.2cm and 0.8cm of b] (c);
    \vertex[below right=1.2cm and 0.8cm of b] (d);
    \diagram*{
        (a) -- [gluon] (b),
        (b) -- [fermion] (c),
        (b) -- [anti fermion] (d),
    };
\end{feynman}
\node[left] at (a) {$A,\mu$};
\node[right] at (c) {$c,j$};
\node[right] at (d) {$b,i$};
\node[right=1.5cm of b] {$=\; -ig_s\,(t^A)_{cb}\,(\gamma^\alpha)_{ji}$};
\end{tikzpicture}

\vspace{0.7cm}

\begin{tikzpicture}[baseline=(current bounding box.center)]
\begin{feynman}
    \vertex (v);
    \vertex[left=2cm of v] (a);
    \vertex[above right=1.3cm and 0.9cm of v] (b);
    \vertex[below right=1.3cm and 0.9cm of v] (c);

    \diagram*{
        (a) -- [gluon, momentum={$p$}] (v),
        (b) -- [gluon, momentum={$q$}] (v),
        (c) -- [gluon, momentum={$r$}] (v),
    };
\end{feynman}

\node[left] at (a) {$A,\mu$};
\node[above right] at (b) {$B,\nu$};
\node[below right] at (c) {$C,\gamma$};

\node[right=2.3cm of v] {
$\displaystyle
= -g_s\, f^{ABC}
\big[
(p-q)^\gamma g^{\mu\nu}
+ (q-r)^\mu g^{\nu\gamma}
+ (r-p)^\nu g^{\mu\gamma}
\big]
$
};

\end{tikzpicture}


\vspace{0.7cm}
\begin{tikzpicture}[baseline=(current bounding box.center)]
\begin{feynman}
    \vertex (v);
    \vertex[above left=of v]  (A) {$A,\mu$};
    \vertex[below left=of v]  (B) {$B,\nu$};
    \vertex[above right=of v] (C) {$C,\gamma$};
    \vertex[below right=of v] (D) {$D,\delta$};

    \diagram*{
        (v) -- [gluon] (A),
        (v) -- [gluon] (B),
        (v) -- [gluon] (C),
        (v) -- [gluon] (D),
    };
\end{feynman}

\node[right=3.1cm of v] {
$\displaystyle
\begin{aligned}
&= -ig_s^2  \left[f^{EAC}f^{EBD}(g^{\mu\nu}g^{\gamma\delta}-g^{\mu\delta}g^{\nu\gamma})\right.\\
&\qquad\quad + f^{EAD}f^{EBC}(g^{\mu\nu}g^{\gamma\delta}-g^{\mu\gamma}g^{\nu\delta})\\
&\qquad\quad \left.+ f^{EAB}f^{ECD}(g^{\mu\gamma}g^{\nu\delta}-g^{\mu\delta}g^{\nu\gamma})\right]
\end{aligned}
$
};
\end{tikzpicture}


\caption{Feynman rules for the quark--gluon vertex, the three--gluon vertex, and the four--gluon vertex.}
\label{fig:QCD-Vertex-Rules}
\end{figure}

\reference{~\cite{Ellis:1996mzs}}
\newpage

\section{Perturbation Theory}
\label{sec:pert-theory}
With the QCD Lagrangian and corresponding Feynman rules established, we can now outline how calculations are performed in practice. Using the propagators and interaction vertices shown in Figs.~\ref{fig:QCD-Propagator-Rules} and~\ref{fig:QCD-Vertex-Rules}, Feynman diagrams can be constructed. These diagrams represent the \emph{scattering (probability) amplitude} $\mathcal{M}$ for a given process and can be evaluated using the Feynman rules. \\ \\
From the structure of the interaction vertices, it is evident that either the emission of an additional gluon (a real correction) or the appearance of a loop (a virtual correction) within a Feynman diagram introduces additional powers of the strong coupling $\alpha_s$. This indicates that the scattering amplitude admits a perturbative expansion,
\begin{equation}
\label{eq:M-pert-theory}
    \mathcal{M}
    = \mathcal{M}^{(0)}
    + \alpha_s\,\mathcal{M}^{(1)}
    + \alpha_s^2\,\mathcal{M}^{(2)}
    + \dots\,,
\end{equation}
where $\mathcal{M}^{(0)}$ denotes the tree-level (classical) contribution, commonly referred to as the leading-order (LO) or Born term. \question{I am wondering if here if there should be a mention of the leading order term being the first non-zero term?}
 The terms $\mathcal{M}^{(1)},\mathcal{M}^{(2)},\dots$ represent higher-order (quantum) corrections known as next-to-leading order (NLO), next-to-next-to-leading order (NNLO), and so on.  \\ \\ Since $\alpha_s$ becomes small at high energies (as discussed in the next section), higher-order contributions are increasingly suppressed in this regime. As a result, the perturbative expansion can be well approximated by retaining only the first few terms, yielding a \emph{fixed-order} (FO) prediction. This approach is known as perturbation theory, and its application to QCD is referred to as perturbative QCD (pQCD). Throughout this thesis, calculations are performed within the framework of pQCD. \\
\reference{none} \\ 

\section{The running of the strong coupling $\alpha_s$}
\label{sec:running-coupling}
As noted in the previous section, pQCD relies on the strong coupling becoming small at high energies. This behaviour reflects the fact that $\alpha_s$ is not a fixed constant but instead \emph{runs} with the renormalisation scale $\mu_R$ -- an unphysical scale introduced by the renormalisation procedure, which will be discussed in the following section. The scale dependence of the coupling is governed by the renormalisation group equation
\begin{equation}
\label{eq:coupling-DE}
    \mu_R^2 \frac{\partial \alpha_s(\mu_R^2)}{\partial \mu_R^2} = \beta(\alpha_s)\,.
\end{equation}
The QCD $\beta$-function admits a perturbative expansion of the form
\begin{equation}
    \beta(\alpha_s) = -\!\left(\beta_0\,\alpha_s^2 + \beta_1\,\alpha_s^3 + \beta_2\,\alpha_s^4 + \cdots\right),
\end{equation}
with coefficients 
\begin{equation}
\begin{aligned}
    \beta_0 &= \frac{11 C_A - 4\,T_R\,n_f}{12\pi}\\[1em]
    \beta_1 &= \frac{17 C_A^2 - 5\,C_A\,n_f - 3\,C_F\,n_f}{2\pi\,(11 C_A - 2 n_f)}\\[1em]
    \beta_2 &= \frac{2857 C_A^3 + (54 C_F^2 - 615 C_F C_A - 1415 C_A^2)n_f + (66 C_F + 79 C_A)n_f^2}{288\pi^2\,(11 C_A - 2 n_f)}\,.
\end{aligned}
\end{equation}
It is worth noting that only $\beta_0$ and $\beta_1$ are renormalisation–scheme independent; the coefficients $\beta_2$ and higher depend on the choice of renormalisation scheme, taken here to be the $\overline{\mathrm{MS}}$ scheme (discussed later). \add{A reference for $\overline{\mathrm{MS}}$.} Solving Eq.~\eqref{eq:coupling-DE} at leading order, using a reference scale $Q$, yields
\begin{equation}
\label{eq:scale-dependence}
    \alpha_s(\mu_R^2)
    = \frac{\alpha_s(Q^2)}
    {1 + \alpha_s(Q^2)\,\beta_0\,\ln\!\left(\tfrac{\mu_R^2}{Q^2}\right)}\, .
\end{equation}
From the sign of $\beta_0$ and the form of Eq.~\eqref{eq:scale-dependence}, the scale dependence of the strong coupling becomes apparent. As $\mu_R \to \infty$, the coupling decreases, and quarks and gluons behave increasingly like free particles. This behaviour is known as \emph{asymptotic freedom}, a regime in which pQCD provides an excellent approximation. Conversely, as $\mu_R$ becomes small and approaches a scale known as $\Lambda_{\mathrm{QCD}}$, the coupling grows large, signalling the breakdown of perturbation theory and the emergence of a strongly interacting regime in which quarks and gluons are confined into hadrons. This phenomenon is known as \emph{confinement}. \question{Is this strictly speaking true of should I be saying $\mu_R\rightarrow 0$ instead here?}\\\\
The running of the coupling has also been confirmed experimentally, as illustrated in Fig.~\ref{fig:strong-coupling}. Throughout this thesis, we will assume that we are working in the high-energy regime, thereby exploiting asymptotic freedom.  \\ \\
\begin{figure}[h]
    \centering
    \includegraphics[width=0.75\linewidth]{figures/strong_coupling.pdf}
    \caption{Measurements of $\alpha_s$ as a function of the energy scale $Q$, taken from Ref.~\cite{ParticleDataGroup:2024cfk}.}
    \label{fig:strong-coupling}
\end{figure}
\reference{~\cite{Ellis:1996mzs, Campbell:2017hsr}}
\section{Divergences}
\label{sec:divergences}
{\color{RoyalBlue} While perturbation theory is highly effective at high energies, it also introduces
certain problems in the form of divergences. In QCD these come in two forms:
\begin{itemize}
    \item \textbf{Ultraviolet (UV):} arise in loop diagrams when the momentum of the
    virtual particle becomes large.
    \item \textbf{Infrared (IR):} arise when a gluon emission becomes soft (low energy) or
    collinear to a hard (high energy) parton.
\end{itemize}
}
\subsection*{UV Divergences}
{\color{RoyalBlue}
Let us begin by discussing UV divergences. One Feynman rule not shown in
Figs.~\ref{fig:QCD-Propagator-Rules} and~\ref{fig:QCD-Vertex-Rules} is the rule associated
with loop diagrams. Each closed loop contributes an integration of the form
\begin{equation}
    \mathrm{loop} \;\longrightarrow\; \frac{d^D\ell}{(2\pi)^D}\,,
\end{equation}
where $\ell$ is the loop momentum. In other words, we must integrate over the momentum
carried by the virtual particle running inside the loop.
To understand how a UV divergence arises, consider the fermion bubble diagram shown in
Fig.~\ref{fig:fermion-bubble}.
\begin{figure}[h]
\centering
\begin{tikzpicture}[scale=2]
\begin{feynman}
    \vertex (a) at (-2,0) {$\mu$, a};
    \vertex (b) at ( 2,0) {$\nu$, b};
    \vertex (c) at (-0.8,0);
    \vertex (d) at ( 0.8,0);

    \diagram*{
        (a) -- [gluon, momentum=$p$] (c),
        (d) -- [gluon, momentum=$p$] (b),
        (c) -- [fermion, half left, looseness=1.5, momentum=$p+q$] (d),
        (d) -- [fermion, half left, looseness=1.5, momentum=$q$] (c),
    };
\end{feynman}
\end{tikzpicture}
\caption{The fermion bubble.}
\label{fig:fermion-bubble}
\end{figure}
Taking the quarks to be massless (as we do throughout this thesis) and applying the Feynman rules, the diagram leads to an integral of the form
\begin{equation}
    I = \int d^4 q \, \frac{N^{\mu\nu}}{q^2 (p+q)^2}\,,
\end{equation}
where $N^{\mu\nu}$ denotes a numerator whose explicit form is not required for the present
discussion. For large loop momenta, the integral behaves as
\begin{equation}
    I \sim \lim_{\Lambda \to \infty} \int_0^\Lambda \frac{dq}{q}
      \sim \lim_{\Lambda \to \infty} \ln \Lambda\,,
\end{equation}
revealing a UV divergence.
We cannot allow divergences to remain in the theory, so we need a way to handle them. This is done through a procedure known as regularisation. Although several regularisation procedures  exist, in this thesis we use \emph{dimensional regularisation}.  This method
preserves both Lorentz and gauge invariance, and as we will see later can also be used
to treat IR divergences. The idea behind dimensional regularisation is to promote the space-time dimension to $d = 4 - 2\varepsilon$, with $\varepsilon < 0$, so that, instead of logarithmic divergences, the integrals develop poles in $\varepsilon$. \\ \\
Once the poles have been isolated, the next step is to remove them. This is achieved by
redefining the parameters of the theory and introducing counterterms that cancel the
divergent pieces. This procedure is known as renormalisation. Throughout this thesis we
work in the $\overline{\mathrm{MS}}$ renormalisation scheme, in which the counterterms
subtract not only the $1/\varepsilon$ poles but also the accompanying constants
$\ln 4\pi$ and $\gamma_E$ (the Euler–Mascheroni constant, $\gamma_E\approx0.5772$) that arise through dimensional regularisation. The renormalisation procedure introduces the renormalisation scale $\mu_R$, and the strong coupling $\alpha_s$ acquires a dependence on this, as discussed in the previous section.
}\\ \\
{\color{orange} Most of what can be found here is in Peskin\&Schroeder.}


\subsection*{IR Divergences}
{\color{RoyalBlue}
Turning our attention to IR divergences, let us consider the simplest QCD process: $e^+e^-\rightarrow \mathrm{hadrons}$. To calculate the LO cross-section, we use the
Feynman rules to evaluate the partonic process $e^+e^- \rightarrow q\bar{q}$, whose
Feynman diagram is shown in Fig.~\ref{fig:photon-qq}. 
\begin{figure}[h]
    \centering
    \begin{tikzpicture}
        \begin{feynman}

            \vertex (e1) at (0,1) {\(e^-\)};
            \vertex (e2) at (0,-1) {\(e^+\)};

            \vertex at (1.5,0) (a);

           \vertex (b) at (3.5,0);

           \vertex (f1) at (4.8,1) {\(q\)};
            \vertex (f2) at (4.8,-1) {\(\overline{q}\)};
            \diagram* {
                (e1) -- [fermion] (a) -- [fermion] (e2),
                (a)  -- [photon, edge label=\(\gamma^*\)] (b),
                (b)  -- [fermion] (f1),
                (b)  -- [anti fermion] (f2),
            };

        \end{feynman}
    \end{tikzpicture}
    \caption{Feynman diagram of \(e^- e^+\) annihilation into a virtual \(\gamma^*\) producing a \(q\bar{q}\) pair.}
    \label{fig:photon-qq}
\end{figure}
This approach is justified by quark-hadron duality: in the high-energy regime, the hadronic cross-section can be approximated sufficiently well by the partonic one calculated in perturbation theory \footnote{There are, however, known exceptions to this approximation; see Ref.~\cite{Shifman:2000jv}.}. \\ \\
{\color{red} (I have a couple of questions used here:
\begin{enumerate}
	\item Quark-hardon duality was initially introduced by Poggio, Quinn and Weinberg - should this 	be cited here?
	\item Is a better option to cite~\cite{Azimov:1984np} - local parton hadron duality - this was cited 
		in Ryan Wood's thesis, and so was wondering if this is a more suitable citation? I am thinking that I need a citation for this for quark-hadron duality itself along with the exceptions to it in Ref.~\cite{Shifman:2000jv}.
	\item I do not need to explain the exact `exceptions' laid out by~\cite{Shifman:2000jv}?
		\item Quark-hadron duality refers to \emph{inclusive} cross-sections - do I need to be 		explicit about this here?
\end{enumerate} 
)} \\ \\
We find that the LO cross-section is
\begin{equation}
	\sigma_{\mathrm{LO}}=\frac{4\pi\alpha^2}{s}N_c\sum_fQ_f^2\,,
\end{equation}
where $Q_f$ is the electric charge of quark flavour $f$, $N_c = 3$ is the number of quark colours, and $\alpha$ is the QED coupling constant. With the LO cross section established, we proceed to the NLO corrections in order to gain insight into the structure of IR divergences. At this order the cross section receives contributions from three diagrams: two real-emission diagrams and one virtual diagram, shown in Figs.~\ref{fig:photon-qqg} and~\ref{fig:photon-qq(g)}, respectively.
\begin{figure}[h]
\centering
\begin{minipage}{0.45\textwidth}
    \centering
    \begin{tikzpicture}
    \begin{feynman}

        \vertex (e1) at (0,1) {\(e^-\)};
        \vertex (e2) at (0,-1) {\(e^+\)};

        \vertex (a) at (1.5,0);

        \vertex (b) at (3.5,0);

        \vertex (f1) at (4.8,1) {\(q\)};
        \vertex (f2) at (4.8,-1) {\(\overline{q}\)};

        \vertex at ($(b)!0.5!(f1)$) (m1);

        \vertex at ($(m1)+(1.0,0)$) (g1) {\(g\)};

        \diagram*{
            (e1) -- [fermion] (a) -- [fermion] (e2),
            (a) -- [photon, edge label=\(\gamma^*\)] (b),

            (b) -- [fermion] (m1) -- [fermion] (f1),
            (b) -- [anti fermion] (f2),

            (m1) -- [gluon] (g1),
        };

    \end{feynman}
    \end{tikzpicture}

\end{minipage}
\hfill
\begin{minipage}{0.45\textwidth}
    \centering
    \begin{tikzpicture}
    \begin{feynman}

        \vertex (e1) at (0,1) {\(e^-\)};
        \vertex (e2) at (0,-1) {\(e^+\)};

        \vertex (a) at (1.5,0);

        \vertex (b) at (3.5,0);

        \vertex (f1) at (4.8,1) {\(q\)};
        \vertex (f2) at (4.8,-1) {\(\overline{q}\)};

        \vertex at ($(b)!0.5!(f2)$) (m2);

        \vertex at ($(m2)+(1.0,0)$) (g2) {\(g\)};

        \diagram*{
            (e1) -- [fermion] (a) -- [fermion] (e2),
            (a) -- [photon, edge label=\(\gamma^*\)] (b),

            (b) -- [fermion] (f1),
            (b) -- [anti fermion] (m2) -- [anti fermion] (f2),

            (m2) -- [gluon] (g2),
        };

    \end{feynman}
    \end{tikzpicture}

\end{minipage}
\caption{Two real-emission diagrams for \(e^- e^+ \to \gamma^* \to q\bar{q}g\): 
gluon emitted from the upper quark leg (left) and from the lower antiquark leg (right).}
\label{fig:photon-qqg}
\end{figure}

\begin{figure}[h]
    \centering
    \begin{tikzpicture}
        \begin{feynman}
            \vertex (e1) at (0,1)  {\(e^-\)};
            \vertex (e2) at (0,-1) {\(e^+\)};
            
  	    \vertex (a) at (1.5,0);
	    \vertex (b) at (3.5,0);
	    
        \vertex (f1) at (4.8,1) {\(q\)};
        \vertex (f2) at (4.8,-1) {\(\overline{q}\)};
            \vertex at ($(b)!0.5!(f1)$) (mid1);
            \vertex at ($(b)!0.5!(f2)$) (mid2);

            \diagram*{
                (e1) -- [fermion] (a) -- [fermion] (e2),
                (a)  -- [photon, edge label=\(\gamma^*\)] (b),
                (b)  -- [fermion]      (f1),
                (b)  -- [anti fermion] (f2),
                (mid1) -- [gluon] (mid2),
            };
        \end{feynman}
    \end{tikzpicture}
    \caption{NLO virtual diagram for \(e^- e^+ \!\rightarrow \gamma^* \!\rightarrow q\bar{q}\).}
    \label{fig:photon-qq(g)}
\end{figure}
We find that the cross-section for the real contributions in Fig.~\ref{fig:photon-qqg} is
\begin{equation}
\label{eq:qqg-real}
	\sigma^{q\bar{q}g}=\sigma_{\mathrm{LO}}\,\,C_F\,\frac{\alpha_s}{2\pi}\int_0^1dx_q\int_{1-x_q}^{1}dx_{\bar{q}}\,\,\frac{x_q^2+x_{\bar{q}}^2}{(1-x_q)(1-x_{\bar{q}})}\,,
\end{equation}
where the energy fractions are defined as 
\[
    x_i = \frac{2 p_i \cdot p}{s}\,, \qquad p=(p_q+p_{\bar{q}}+p_g)\,,
    \]
 where $s=(p_{e^+}+p_{e^-})^2$ is the centre-of-mass (CoM) energy squared. 
From the integrand we see that divergences arise in the following limits:
\begin{enumerate}
    \item \(x_q \to 1\) \emph{and} \(x_{\bar{q}} \to 1\): soft emission (\(E_g \to 0\));
    \item \(x_q \to 1\) \emph{or} \(x_{\bar{q}} \to 1\): collinear emission 
          (\(\theta_{\bar{q}g},\, \theta_{qg} \to 0\)).
\end{enumerate}
These are infrared and collinear singularities, collectively referred to as IR (or IRC) divergences. Using dimensional regularisation to isolate them, Eq.~\ref{eq:qqg-real} becomes
\begin{equation}
    \sigma^{q\bar{q}g}
    = \sigma_{\mathrm{LO}}\, \frac{\alpha_s}{2\pi}\,
      \frac{C_F}{\Gamma(1-\varepsilon)}
      \left(\frac{4\pi\mu_R^2}{s}\right)^{\varepsilon}
      \left(
          \frac{2}{\varepsilon^2}
        + \frac{3}{\varepsilon}
        - \pi^2
        + 8
        + \mathcal{O}(\varepsilon)
      \right).
\end{equation}
The virtual correction produces poles with precisely the same structure but opposite sign:
\begin{equation}
    \sigma^{q\bar{q}(g)}
    = \sigma_{\mathrm{LO}}\, \frac{\alpha_s}{2\pi}\,
      \frac{C_F}{\Gamma(1-\varepsilon)}
      \left(\frac{4\pi\mu_R^2}{s}\right)^{\varepsilon}
      \left(
         -\frac{2}{\varepsilon^2}
         -\frac{3}{\varepsilon}
         + \pi^2
         - 8
         + \mathcal{O}(\varepsilon)
      \right).
\end{equation}
Adding the real and virtual contributions gives the finite NLO correction,
\[
    \sigma_{\mathrm{NLO}}
    = \sigma^{q\bar{q}g} + \sigma^{q\bar{q}(g)}
    = \sigma_{\mathrm{LO}}\, \frac{\alpha_s}{2\pi}\,
      \frac{C_F}{\Gamma(1-\varepsilon)}
      \left(\frac{4\pi\mu_R^2}{s}\right)^{\varepsilon}
      \left(\frac{3}{2} + \mathcal{O}(\varepsilon)\right),
\]
where taking the limit \(\varepsilon \to 0\) yields
\begin{equation}
    \sigma_{\mathrm{NLO}}
    = \frac{\alpha_s}{2\pi}\, \frac{3}{2}\, C_F\, \sigma_{\mathrm{LO}}\,.
\end{equation}
As a result, the total cross section up to NLO is
\begin{equation}
    \sigma
    = \sigma_{\mathrm{LO}}
      \left(1 + \frac{\alpha_s}{2\pi}\,\frac{3}{2} C_F 
      + \mathcal{O}(\alpha_s^2)\right).
\end{equation}
The cancellation of singularities between the real and virtual diagrams is not
unexpected—it is a direct consequence of the \emph{KLN theorem}~\cite{Kinoshita:1962ur,
Lee:1964is}. This theorem guarantees that IR divergences cancel between real and virtual contributions order by order in perturbation theory for any process, provided one sums over all initial and final states.} \\ \\
{\color{orange} Done with Luisoni \& Marzani, Schwartz and the pink book \& thesis' and powerpoint. For UV divergences, no formal references right at this second.}


\section{The $2\rightarrow m$ partonic cross-section}
\label{sec:2-to-m-x-sec}
{\color{RoyalBlue}
{\color{red} (Would it be better to have this section in an earlier section, such that after discussing IR divergences we can jump straight into soft and collinear factorisation?)} \\ \\
In the previous section, we computed the NLO cross-section for a specific process and saw explicitly how soft and collinear singularities arise. To understand this behaviour in a more general context, it is useful to step back and consider the structure of a generic partonic cross section. For a $2\rightarrow m$ hard scattering process,
\[
    p_a + p_b \rightarrow p_1 + \cdots + p_m\,,
\]
the parton-level cross section is given by
\begin{equation}
\label{eq:2-to-m-xsec}
    \sigma= \frac{1}{2s}\int d\Phi^{(m)}(p_1,\ldots,p_m)\,|\mathcal{M}_m(p_a,p_b; p_1,\ldots,p_m)|^2\,F_J^{(m)}(p_1,\ldots,p_m)\,,
\end{equation}
{\color{red}(I feel as though $\sigma$ should be $\hat{\sigma}$?)}
where the components of this expression are:
\begin{itemize}
	\item $d\Phi^{(m)}$ is the $m$-parton phase space,
\begin{equation}
	d\Phi^{(m)}=(2\pi)^4\delta^{(4)}\left(p_a+p_b-\sum_{i=1}^{m}p_i\right)\prod_{i=1}^{m}\,\frac{d^3p_i}{(2\pi)^3\,2E_i}\,.
\end{equation}
\item $\mathcal{M}_m$ is the scattering amplitude for the $2\!\rightarrow\! m$ process. It admits the perturbative expansion shown in Eq.~\ref{eq:M-pert-theory} and its modulus
    squared is written as
    \[
        |\mathcal{M}_m|^2 = \langle \mathcal{M}_m | \mathcal{M}_m \rangle .
    \]
    {\color{red} (Do I need to explain the bra/ket and the basis these take?)}
\item $F_J^{(m)}$ is the \emph{jet function}, which specifies the observable computed.
 Depending on the choice of $\Theta$–functions, $\delta$–functions, or numerical
factors, it can describe: cross-sections, differential cross sections or expectation values respectively. For $F_J^{(m)} = 1$, Eq.~\ref{eq:2-to-m-xsec} reduces to the total cross-section, as in the
$e^+e^-$ example of the previous section - more will be discussed on this later.
\end{itemize}
{\color{red} (I need a reference for this section - for this equation in particular.)}
}

\section{Soft and collinear factorisation of $|\mathcal{M}_m|^2$}
{\color{RoyalBlue}
With the general form of the partonic cross section established, we now turn to the behaviour
of the squared scattering amplitude $|\mathcal{M}_m|^2$.  
In the soft and collinear limits, this quantity factorises into a remarkably simple and universal form. The relevant expressions are summarised below.

\subsection*{Soft limit}
Suppressing the initial–state momenta $(p_a,p_b)$ for brevity, the squared matrix
element for the emission of an additional soft parton of momentum $q$ behaves as
\begin{equation}
    |\mathcal{M}_{m+1}^{(0)}(q,p_1,\ldots,p_m)|^2
    \simeq
    -4\pi\alpha_s
    \sum_{i,j=1}^{m}
        S_{ij}(q)\,
        \big\langle
            \mathcal{M}_m^{(0)}(p_1,\ldots,p_m)
        \big|
            \mathbf{T}_i \cdot \mathbf{T}_j
        \big|
            \mathcal{M}_m^{(0)}(p_1,\ldots,p_m)
        \big\rangle,
\end{equation}
where the colour operators satisfy
\begin{equation}
\begin{aligned}
    \mathbf{T}_i \cdot \mathbf{T}_j &= \mathbf{T}_j \cdot \mathbf{T}_i \qquad (i\neq j),\\[4pt]
    \mathbf{T}_i^2 &= C_i \qquad (C_i = C_F \text{ for a quark, } C_A \text{ for a gluon}),
\end{aligned}
\end{equation}
and the eikonal factor
\[
    S_{ij}(q) = \frac{p_i \cdot p_j}{(p_i \cdot q)(p_j \cdot q)}
\]
encodes the soft dynamics of QCD.

\subsection*{Collinear limit}

When two partons $i$ and $j$ become collinear, the squared matrix element factorises as
\begin{equation} 
|\mathcal{M}_{m+1}^{(0)}(p_1,\cdots p_i, p_j, \cdots p_{m+1})|^2\simeq4\pi\alpha_s\bra{\mathcal{M}_{m}^{(0)}(p_1,\cdots p_{ij}, \cdots p_m)}\hat{\mathbf{P}}^{(0)}_{ij}(z,k_\perp;\epsilon)\ket{\mathcal{M}_{m}^{(0)}(p_1,\cdots p_{ij}, \cdots p_m)}\,.
\end{equation}
The operators $\hat{\mathbf{P}}^{(0)}_{ij}$ reduce to the usual Altarelli-Parisi splitting functions {\color{red} (Unsure if this is correct i.e. if they `reduce' to it?)}. In explicit form:
\begin{equation}
\begin{aligned}
\label{eq:AP-functions}
	&\hat{P}_{q\bar{q}}^{(0)}(z,k_\perp;\epsilon)=\hat{P}_{\bar{q}q}^{(0)}(z,k_\perp;\epsilon)=T_R\left[-g^{\mu\nu}+4z(1-z)\frac{k_\perp^{\mu}k_\perp^{\nu}}{k_\perp^2}\right]\\
	&\hat{P}_{qg}^{(0)}(z,k_\perp;\epsilon)=\hat{P}_{{\bar{q}}g}^{(0)}(z,k_\perp;\epsilon)=\delta^{ss'}C_F\left[\frac{1+z^2}{1-z}-\epsilon(1-z)\right]\\
	&\hat{P}_{gq}^{(0)}(z,k_\perp;\epsilon)=\hat{P}_{g{\bar{q}}}^{(0)}(z,k_\perp;\epsilon)=\delta^{ss'}C_F\left[\frac{1+(1-z)^2}{z}-\epsilon z\right]\\
	&\hat{P}_{gg}^{(0)}(z,k_\perp;\epsilon)=2C_A\left[-g^{\mu\nu}\left(\frac{z}{1-z}+\frac{1-z}{2}\right)-2(1-\epsilon)z(1-z)\frac{k_\perp^{\mu}k_\perp^{\nu}}{k_\perp^2}\right]\,,
\end{aligned}
\end{equation}
where $\varepsilon$ is the usual dimensional-regularisation parameter and $s,s'$ denote the spins of the two partons.\\ \\
To obtain these expressions, it is convenient to use the Sudakov decomposition for the
two collinear momenta,
\begin{equation}
\begin{aligned}
    p_i^\mu &= z_i\, p^\mu + k_\perp^\mu
               - \frac{k_\perp^2}{z_1}\,\frac{n^\mu}{2p\cdot n}, \\
    p_j^\mu &= z_j\,p^\mu - k_\perp^\mu
               - \frac{k_\perp^2}{z_j}\,\frac{n^\mu}{2p\cdot n},
\end{aligned}
\end{equation}
where $z_i+z_j=1$ are the longitudinal momentum fractions, $p^\mu$ and $n^\mu$ are light-cone {\color{red}(light-cone or light-like?)} reference vectors, $k_\perp^\mu$ is transverse to both, and the collinear limit corresponds to $k_\perp^\mu \to 0$. \\ \\
{\color{red} (Should there be a discussion of subtraction/slicing here?)} \\ \\
{\color{red} (Need some more references here - particularly for the splitting functions.)} \\ \\
{\color{orange} This section used Luosiani/Marzani paper and Schwartz.}
}

\section{IRC Safety} 
\label{sec:IRC}
{\color{RoyalBlue}
In Sec.~\ref{sec:divergences} we saw, through the explicit NLO calculation of $e^{+}e^{-}\!\to q\bar q g$, that the infrared singularities from the real and virtual  contributions cancel, yielding a finite result. This relies on the fact that the total cross section is an \emph{inclusive} observable, for which 
$F_J^{(m)} = 1$  for any number of final-state particles $m$ (see Sec.~\ref{sec:2-to-m-x-sec}). \\ \\
For more general observables such a cancellation is not guaranteed. To obtain a finite prediction order by order in perturbation theory, the observable must be \emph{infrared and collinear safe} (IRC safe). An IRC safe observable is one whose value is insensitive to the emission of an 
additional soft/collinear parton. This may be expressed in terms of the jet function:
\begin{equation}
\begin{aligned}
F_J^{(m+1)}(p_1,\ldots,p_i,\ldots,p_{m+1})
&\;\longrightarrow\;
F_J^{(m)}(p_1,\ldots,\cancel{p_i},\ldots,p_{m+1})\,,
\qquad &&E_i \to 0, \\[6pt]
F_J^{(m+1)}(p_1,\ldots,p_i,p_j,\ldots,p_{m+1})
&\;\longrightarrow\;
F_J^{(m)}(p_1,\ldots,\cancel{p_i},\cancel{p_j},p_{ij},\ldots,p_{m+1})\,,
\qquad &&p_i \parallel p_j\,,
\end{aligned}
\label{eq:IRC-safety}
\end{equation}
where a diagrammatic representation of IRC safety is shown in Fig.~\ref{fig:IRC_safety}. 
\begin{figure}[h]
\centering
\begin{tikzpicture}[scale=2]
\begin{feynman}

    \vertex (a) at (-1, 0.6) {\(p_a\)};
    \vertex (b) at (-1,-0.6) {\(p_b\)};

    \vertex[blob, minimum size=0.8cm] (H) at (0,0) {\(\mathcal{H}\)};

    \vertex (p1) at (1.5, 1.2) {\(p_1\)};
    \vertex (pj) at (1.5, 0.4) {\(p_j\)};  
    \vertex (pm) at (1.5,-1.0) {\(p_{m}\)};

    \vertex (x) at (0.72, 0.2);
    
    \vertex (pi) at (1.5, 0.65) {\(p_i\)};

    \diagram*{
        (a) -- [fermion] (H) -- [fermion] (p1),
        (b) -- [fermion] (H) -- [fermion] (pm),

        (H) -- [fermion] (pj),
        
    };


    \draw[gluon] (x) -- (pi);


    \node at (1.5, 0.9) {\(\cdots\)};
    \node at (1.5, -0.3) {\(\cdots\)};
\node[
    draw=black!30,
    fill=gray!5,
    rounded corners=3pt,
    font=\scriptsize,
    inner sep=4pt
] at (3.0,0.50)
{
    \shortstack{$F_J^{(m+1)}(\Phi^{(m+1)}) \;\longrightarrow\; F_J^{(m)}(\Phi^{(m)})$\\[2pt]
                $\text{if } E_i \to 0 \;\text{or}\; p_i \parallel p_j$}
};


\end{feynman}
\end{tikzpicture}
\caption{Diagrammatic representation of a $2\to m$ hard scattering with an additional soft/collinear emission $p_i$. {\color{blue} (Change this-do we need to ad $\Phi^{(m+1)}\neq\Phi^{(m)}$? Also do we need to explain the $\mathcal{H}$ and should it actually be $|\mathcal{M}|^2$)?}}
\label{fig:IRC_safety}
\end{figure}
We emphasise that although Eq.~\ref{eq:IRC-safety} is written for a single additional parton, the same behaviour holds for any number of additional soft/collinear partons. From this definition it follows immediately that inclusive observables with $F_J^{(m)} = 1$ are IRC safe. Other examples include event-shape variables (event-shapes), discussed in Chapter~4. A simple example of an IRC-unsafe observable is the number of final-state partons, which of course changes whenever a soft or collinear emission occurs. \\ \\
Although IRC safety ensures a finite results at every order in perturbation theory, it does not guarantee a well-behaved perturbative expansion. Many \emph{exclusive} observables, despite being IRC safe, receive large logarithmic contributions which originate from emissions in the soft and/or collinear regions. These large corrections can spoil the convergent nature of the perturbative expansion and in such cases one must perform \emph{resummation}, which will be the focus of Chapters~4 and~5. \\ \\
}
{\color{red} I have a few questions on this:
\begin{enumerate}
\item In this sense is the KNL theorem only for inclusive observables? I ask since it states that there will be finite results at every order of perturbation theory if we sum over all initial and final states - I presume this summing over is what makes it an inclusive observable?
\item Checking that exclusive observables are those that give large logarithms i.e. inclusive observables don't get receive these? This is what essentially distinguishes these two types of observables in a resummation sense?
\item I need a reference for this section I think i.e. not a book or review
\end{enumerate}}
{\color{orange} Luoisiani/Marzani, powerpoints, A. Banfi book, Black Book.}


\section{The full picture: high-energy collisions}
Up until this point, the discussion in this thesis has largely focused on physics at the
parton level. In particular, we have relied on concepts such as quark-hadron duality (see Sec.~\ref{sec:divergences}) to relate partonic cross-sections to hadronic observables. We have also worked predominantly in the high-energy regime, where asymptotic freedom (see Sec.~\ref{sec:running-coupling}) ensures that perturbative QCD is applicable. \\ \\
In realistic collider experiments, however, the picture is far more involved. Consider, for
example, a hadronic collision at the LHC. The incoming particles are hadrons opposed to partons, and the final-state particles observed in the detector consists of hadrons produced at energy scales far below those of the hard interaction. As a result, a complete description of a high--energy collision necessarily involves physics beyond the perturbative regime.\\ \\
The aim of this section is to outline the various stages of a high-energy hadronic collision. To illustrate this, we will refer to Fig.~\ref{fig:high-energy-event}, produced by the \textsc{SHERPA} collaboration in Ref.~\cite{Gleisberg:2008ta}, which shows a $t\bar t h$ event produced by a proton-proton collision. {\color{blue} (Potentially discuss what \textsc{SHERPA} is?)}
\begin{figure}[h]
    \centering
    \includegraphics[width=0.75\linewidth]{figures/High-energy-event.jpeg}
    \caption{Pictorial representaion of a $t\bar{t}h$ event produced in a high-energy hadronic collision, taken from Ref.~\cite{Gleisberg:2008ta}}.
    \label{fig:high-energy-event}
\end{figure}
Since the initial collision occurs at high energies and the energy of the system decreases from here until we reach hadronisation scales of far lower energy, it is natural for us to organise the following discussion according to the energy scale $Q$ of the process. 
\subsection*{Hard scattering $(Q \gg \Lambda_{\mathrm{QCD}})$}
At the highest energy scale, the dynamics of the collision are governed by the hard scattering, represented by the central red blob in Fig.~\ref{fig:high-energy-event}. In this process, two partons, denoted $p_i$ and $p_j$, are extracted from the incoming
protons $P_1$ and $P_2$ and undergo a short-distance interaction.\\ \\
The partonic content of the proton is described by parton distribution functions (PDFs), which encode its internal structure. A PDF $f_{i,P_1}(x_1,\mu_F^2)$ gives the probability of finding a parton of type $i$ inside the proton $P_1$ carrying a longitudinal momentum fraction $x_1$, evaluated at the factorisation scale $\mu_F$.  These are non-perturbative functions which are extracted from data and depend on the scale they are probed at. The scale dependence of the PDFs is governed by the Dokshitzer-Gribov-Lipatov-Altarelli-Parisi (DGLAP) equation
\begin{equation}
	\frac{df_{i,P_1}(x_1,\mu_F^2)}{d\mu_F^2}=\sum_{j}\int_{x_1}^1dz\,\frac{\alpha_s}{2\pi}P_{ij}(z,\mu_F^2)f_j\left(\frac{x_1}{z},\mu_F^2\right)\,,
\end{equation}
where the splitting functions $P_{ij}(z)$ admit a perturbative expansion,
\begin{equation}
	P_{ij}(z,\mu_F^2) = P_{ij}^{(0)}(z)+\frac{\alpha_s}{2\pi} P_{ij}^{(1)}(z)+\dots\,.
\end{equation}
At leading order, the \emph{regularised} Altarelli-Parisi (see Eq.~\ref{eq:AP-functions} for unregularised forms) splitting functions are given by 
\begin{equation}
\begin{aligned}
&P_{q\bar{q}}^{(0)}(z)=C_F\left[\frac{1+z^2}{(1-z)_+}+\frac{3}{2}\delta(1-z)\right]\\
&P_{qg}^{(0)}(z)=T_R\left[z^2+(1-z)^2\right]\\
&P_{gq}^{(0)}(z)=C_F\left[\frac{1+(1-z)^2}{z}\right]\\
&P_{gg}^{(0)}(z)=2C_A\left[\frac{z}{(1-z)_+}+\frac{1-z}{z}+x(1-z)\right]+\delta(1-z)\frac{11C_A-4T_Rn_f}{6}\,.
\end{aligned}
\end{equation}
\\ \\
The factorisation scale $\mu_F$ is an arbitrary scale that separates long--distance physics from short--distance dynamics. Emissions with transverse momentum $p_\perp$ below $\mu_F$ are absorbed into the PDFs and treated as part of the hadron structure,while emissions with $p_\perp > \mu_F$ are included in the partonic cross section. \\ \\ 
The hard interaction produces the primary particles of the event, in this case a
$t\bar{t}h$ system, indicated by the smaller red blobs in Fig.~\ref{fig:high-energy-event}. Since the characteristic scale $Q$ is much larger than $\Lambda_{\mathrm{QCD}}$, fixed-order perturbation theory is applicable (see Sec.~\ref{sec:pert-theory}). Predictions at this stage are obtained from matrix-element calculations, typically provided by matrix-element generators. {\color{blue} (Put in some references for examples of matrix element generators?)}

\subsection*{QCD evolution $(Q \gtrsim \Lambda_{\mathrm{QCD}})$}
Following the hard scattering, the coloured partons produced in the event undergo further QCD radiation, illustrated by the red quark and gluon lines in Fig.~\ref{fig:high-energy-event}. This radiation is dominated by soft and collinear emissions, reflecting the infrared structure of QCD discussed in Sec.~\ref{sec:divergences}.
For many observables, these emissions give rise to logarithmically enhanced contributions (see Sec.~\ref{sec:IRC}) which can become large and spoil the convergence of fixed-order perturbation theory. As a result, methods are required to deal with these emissions such that we can work within the perturbative regime. Approaches which are commonly used are:
{\color{green} Go from here please.}
\begin{itemize}
	\item \emph{Parton showers}, which model the
	soft and collinear emissions down to the hadronisation scale. Well-known examples
	include: \textsc{Pythia}, \textsc{Herwig}, and \textsc{PanScales}. Currently parton showers achieve next-to-leading logarithmic (NLL) accuracy for a broad class of observables, with
	ongoing work aimed at extending this to higher logarithmic accuracy. {\color{red} Is it completely numerical? Are these the correct panscales to reference?}
	\item \emph{Resummation}, where logarithmically enhanced contributions from soft and
	collinear radiation are essentially \emph{resummed} to all orders in perturbation theory. Semi-numerical
	resummation frameworks such as \textsc{CAESAR}~\cite{Banfi:2004yd} and \textsc{ARES}~\cite{Banfi:2014sua} fall into this
	category. For $e^+e^-$ observables, resummation is known up to next-to-next-to-leading
	logarithmic (NNLL) accuracy~\cite{Arpino:2019ozn}, while hadronic observables are currently known up to NLL.
	The work presented in Chapter~5 focuses on extending the \textsc{ARES} framework to
	hadronic collisions at NNLL accuracy.
		\item \emph{Effective field theory} approaches, most notably soft-collinear effective
	theory (SCET), which describe soft and collinear degrees of freedom through an
	observable-dependent factorisation formula. In this framework, the relevant dynamics
	are encoded in separate hard, collinear, soft, and jet functions, allowing resummation to
	be performed analytically for a range of observables, in some cases up to $\mathrm{N^3LL}$
	accuracy. {\color{red} (Is SCET I analytical while SCET II requires numerical techniques?)} 
\end{itemize}
The radiation described in this regime connects the short-distance hard interaction to the non-perturbative hadronisation process. As the characteristic scale of the event continues to decrease, perturbative methods eventually break down, and the dynamics become dominated by confinement effects, which are discussed next.

\subsection*{Hadronisation ($Q \sim \Lambda_{\mathrm{QCD}}$)}
As the characteristic scale of process approaches $\lambda_{\mathrm{QCD}}$, perturbation theory breaks down. In this regime, the radiated final-state partons (light green blobs in Fig.~\ref{fig:high-energy-event}) undergo hadronisation and transform into primary hadrons (dark green blobs). These hadrons subsequently decay into the long-lived 
final-state particles (smaller dark green blobs) that are ultimately observed 
in the detector. Experimentally, these particles appear as \emph{jets}: collimated sprays of hadrons. Jet clustering and its implications will be discussed in the next section. Since hadronisation occurs in the non-perturbative regime of QCD, it cannot be calculated from first principles. Instead, we must use numerical models to calculate its effects. {\color{red}(I see that there are a lot of references for what models in the black book and~\cite{Gleisberg:2008ta} - however am unsure if these are out of date now.)} 
\\ \\ 
The three stages described above capture the main features of the QCD dynamics, but several other ingredients contribute to the structure of a realistic hadronic event and must be briefly mentioned. 
First, electromagnetic radiation (shown as yellow lines in Fig.~\ref{fig:high-energy-event}) may occur at any stage of the process. Second, proton remnants may undergo additional semi-hard interactions, giving rise to the \emph{underlying event}, represented by the purple blob in Fig.~\ref{fig:high-energy-event}. These secondary interactions produce further radiation and hadrons, much like the primary hard process, but they lie outside the scope of standard QCD factorisation
theorems and therefore require modelling as discussed in Ref.~\cite{Gleisberg:2008ta}. Finally, before the hard interaction occurs, the incoming partons can radiate, leading to \emph{initial-state radiation}.  This radiation evolves down to the hadronisation scale and contributes to the final-state hadrons (dark green blobs). The light blue ovals in Fig.~\ref{fig:high-energy-event} represent the beam remnants: the fragments of the incoming protons that do not participate in the hard collision.
\subsection*{The QCD Factorisation Formula}
To relate the partonic dynamics discussed so far to experimentally measured hadronic cross sections, we can employ the QCD factorisation formula,
\begin{equation}
\label{eq:factorisation-formula}
	\sigma(P_1,P_2)=\sum_{i,j}\int_0^1dx_1\,dx_2\,f_{i,P_1}(x_1,\mu_F^2)\,f_{j,P_2}(x_2,\mu_F^2)\,\hat{\sigma}_{ij}(\alpha_s(\mu_R^2),\mu_R^2,\mu_F^2)+\mathcal{O}\left(\frac{\Lambda_{\mathrm{QCD}}^2}{Q^2}\right)\,.
\end{equation}
Here $f_{i,P_1}$ and $f_{j,P_2}$ are the PDFs introduced earlier, and the integration runs over the momentum fractions carried by the partons inside the incoming hadrons. The sum over $i,j$ accounts for all possible parton flavours,
while $\hat{\sigma}_{ij}$ is the corresponding partonic cross section. \\ \\
Equation~\ref{eq:factorisation-formula} is referred to as a factorisation formula because it separates the hadronic cross section into two distinct components: the non-perturbative PDFs, which encode the partonic structure of the incoming
hadrons, and the perturbatively calculable partonic cross section $\hat{\sigma}_{ij}$. The term
$\mathcal{O}\!\left(\Lambda_{\mathrm{QCD}}^{2}/Q^{2}\right)$ reflects the fact that
this separation is valid in the high-energy limit.\\ \\
It should be noted that in this thesis we are mainly working within the $(Q \gg \Lambda_{\mathrm{QCD}})$ regime - particularly in Chapter~4 and~5, where we discuss resummation techniques. {\color{red} Do we use this regime for jet algorithms - I don't really think we do?}


{\color{orange}Used pink book, Edinburgh notes, and A Banfi powerpoint recommendation for this mainly.}
  







\section{Jet Physics}
{\color{RoyalBlue}
Jets are among the most frequently observed objects in proton–proton collisions at the LHC. They appear as highly collimated sprays of hadrons resulting from the hadronisation of energetic quarks and gluons produced in high-energy processes~\cite{Banfi:2016yyq}. When a hard parton is created in a short-distance interaction, it radiates gluons in a similarly collimated pattern along its initial direction. These gluons can in turn branch into further gluons or quark–antiquark pairs~\cite{Schwartz:2014sze}. As the radiation proceeds, the typical energy scale of the process decreases and the interactions between the partons become stronger, until they reach an energy of order $\Lambda_{\mathrm{QCD}}$. At this stage, the partons cluster together to form the final-state hadrons observed in the detector. These hadrons tend to group into fairly energetic, collimated clumps - known as \emph{jets}~\cite{Schwartz:2014sze, Campbell:2017hsr, Banfi:2016yyq}.
}




\subsection{Jet Definition}
{\color{RoyalBlue}
As described above, jets are complex, messy, and inherently ambiguous objects. To be able to describe them in a well-defined and reproducible way, we must introduce a \emph{jet definition}, which consists of the following components~\cite{Salam:2010nqg}:
\begin{enumerate}
\item a \emph{jet algorithm}: a systematic procedure for grouping final-state particles into jets and determining the number of jets present in an event, i.e.\ a set of rules that cluster particles into a given jet;
\item the algorithm's \emph{parameters}: these specify the proximity required between two particles for them to be recombined into a single entity belonging to the same jet;
\item a \emph{recombination scheme}: this defines how the momenta of two particles are combined to form the momentum of the new clustered particle — the most commonly used being the $E$-scheme, , which is a four-momentum sum, i.e.\ $p_i^\mu + p_j^\mu = p_k^\mu$.
\end{enumerate}
In the next subsection, we discuss the jet algorithm and its input parameters in more detail. }

\subsection{Jet Algorithm}
{\color{RoyalBlue}
In 1990, there was significant discussion within the high-energy physics community about how jets should be properly defined. To address this, a group of physicists formulated a set of requirements that a well-defined jet algorithm should satisfy, outlined in a document known as the \emph{Snowmass Accord}~\cite{Huth:1990mi}. The criteria are as follows:
\begin{enumerate}
    \item Simple to implement in an experimental analysis;
    \item Simple to implement in a theoretical calculation;
    \item Defined at any order in perturbation theory;
    \item Yields a finite cross section at any order in perturbation theory;
    \item Yields a cross section that is relatively insensitive to hadronisation effects.
\end{enumerate}
IRC safety ensures that jets defined at the detector, hadron, and parton levels are essentially equivalent, thereby satisfying the final three requirements outlined above. For IRC-safe observables, the effects of hadronisation are suppressed by inverse powers of the energy scale used in the process. Consequently, at higher energies, the correspondence between hadronic and partonic observables becomes increasingly accurate~\cite{Banfi:2016yyq}.
Overall, these requirements allow us to to regard jets as well-defined objects whose properties can be either determined by their constituent hadrons or underlying partons. This enables direct comparison between experimental measurements and theoretical calculations at the parton level. \\ \\
A wide variety of jet algorithms exist, each with its own advantages and limitations. The choice of algorithm depends on the specific physics question being addressed and the type of information one seeks to extract from an event. In general, jet algorithms can be divided into two broad classes: \emph{sequential recombination algorithms} and \emph{cone algorithms}~\cite{Banfi:2016yyq},~\cite{Salam:2010nqg}.
\begin{itemize}
    \item \textbf{Cone algorithms:} These group particles within a fixed radius $R$ in $(\eta,\phi)$ space, such that the particles whose transverse energy deposits fall inside the circular region are clustered into a jet. In three dimensions, these regions appear as cones. Different implementations vary in how they search for and define these cones. Cone algorithms are often described as \emph{top-down} algorithms.

    \item \textbf{Sequential recombination algorithms:} These iteratively cluster the closest particles according to a chosen distance measure until a stopping criterion is reached. Variations arise from different definitions of the distance measure (e.g.\ relative transverse momentum or angular separation between particles) and from the choice of stopping condition. Sequential recombination algorithms are often referred to as \emph{bottom-up} algorithms.
\end{itemize}
In this thesis, we focus exclusively on sequential recombination algorithms. The following sections discuss several of these algorithms to provide background for the new DIS algorithm introduced in Chapter~\ref{chapter: DIS}. For a comprehensive review of jets and jet algorithms across different processes, see Ref.~\cite{Salam:2010nqg}.
}

\subsection{Jet Algorithms for $e^+e^-$ Collisions}
\label{subsec: e+e-}
{\color{RoyalBlue}
Jet algorithms for electron–positron collisions follow a common iterative clustering procedure. Each pair of particles is assigned a \emph{distance measure}, $d_{ij}$, which determines how closely they are related. The closest pair is iteratively recombined until all remaining objects are separated by more than the chosen \emph{resolution parameter}, $d_{\mathrm{cut}}$.
This framework forms the basis of the JADE and Durham algorithms, while the Cambridge algorithm extends it by introducing an additional \emph{ordering variable}, $v_{ij}$, as discussed in the following sections.

\subsubsection{The JADE algorithm}
The first sequential recombination algorithm was introduced by the JADE Collaboration in 1988~\cite{JADE:1988xlj} and is defined as follows:
\begin{enumerate}
    \item \textbf{Define a resolution parameter} $d_{\mathrm{cut}}$.
    \item \textbf{Compute the distance measure} for every pair of particles $i$ and $j$:
    \begin{equation}
        \label{eq:yij-jade}
        d^{\mathrm{JADE}}_{ij} = \frac{2E_iE_j(1 - \cos\theta_{ij})}{Q^2}\,,
    \end{equation}
    where $Q$ is the total centre-of-mass energy, $E_i$ is the energy of particle $i$, and $\theta_{ij}$ is the angle between particles $i$ and $j$.
    \item \textbf{Identify the smallest $d_{ij}$ value.} If this minimum satisfies $d_{ij} < d_{\mathrm{cut}}$, the two particles are recombined into a pseudo-particle according to the chosen recombination scheme.
    \item \textbf{Repeat the procedure} from step 2 until all remaining pairs satisfy $d_{ij} > d_{\mathrm{cut}}$. The remaining objects are then defined as jets.
\end{enumerate}
A drawback of the JADE algorithm is its tendency to cluster soft pairs of gluons that are widely separated in angle. This can lead to situations where a hard parton is incorrectly merged with a soft gluon that was not emitted from it, as shown in Fig.~\ref{fig:JADE}. Physically speaking, a jet should constitute to a hard parton together with its associated radiation, so such clustering is undesirable. This issue is resolved by modifying the distance measure, leading to the $k_\perp$ algorithm.
\begin{figure}[h]
    \centering
    \includegraphics[width=0.8\linewidth]{figures/JADE.jpeg}
    \caption{JADE algorithm issues from Ref.~\cite{Banfi:2016yyq}}
    \label{fig:JADE}
\end{figure}
}


\subsubsection{ The $k_\perp$ (Durham) algorithm}
{\color{RoyalBlue}
Also known as the Durham algorithm~\cite{Catani:1991hj}, this approach follows the same procedure as JADE but employs a modified distance measure:
\begin{equation}
	 \label{eq:yij-De}
    	d^{\mathrm{Durham}}_{ij}=\frac{2(1-\cos\theta_{ij})}{Q^2}\min(E_i^2,E_j^2)\,.
\end{equation} 
At small angles, the numerator can be approximated to $(\min(E_i,E_j)\theta_{ij})^2$, which corresponds to the squared relative transverse momentum of particle $i$ with respect to particle $j$ (for $E_i < E_j$). The use of the $\min$ function ensures that soft emissions, widely separated in angle, have a larger distance measure than those corresponding to a hard parton radiating a nearby soft gluon. This modification prevents unphysical clustering of unrelated soft particles and produces jets that more accurately reflect the underlying partonic structure.
}
\subsubsection{The Cambridge algorithm}
{\color{RoyalBlue}
The Cambridge algorithm, first introduced in Ref.~\cite{Dokshitzer:1997in}, extends the Durham algorithm by employing angular ordering rather than transverse-momentum ordering. Unlike the previous two algorithms, it introduces an additional \emph{ordering variable}, $v_{ij}$, used alongside the distance measure $d^{\mathrm{Durham}}_{ij}$ from the Durham algorithm:
\begin{enumerate}
    \item \textbf{If only one particle remains,} stop the clustering and define this object as a jet.
    \item \textbf{Find the smallest $v_{ij}$} among all pairs of particles:
    \begin{equation}
        \label{eq:vij-cambridge}
        v_{ij} = 2(1 - \cos\theta_{ij})\,.
    \end{equation}
    \item \textbf{Identify the corresponding $d_{ij}$ value.} If $d_{ij} < d_{\mathrm{cut}}$, recombine the particles as in the Durham algorithm and return to step 1.
    \item \textbf{Otherwise,} remove the less energetic particle, label it as a jet, and return to step 1.
\end{enumerate}
This clustering procedure effectively reconstructs the sequence of gluon emissions in reverse, which typically occur at progressively smaller angles~\cite{Banfi:2016yyq}. As a result, the Cambridge algorithm is a better algorithm for the resolution of jet substructure and for reducing non- perturbative effects, which occur since emissions widely separated in angles are emitted independently from the hard legs. 
}
{\color{red} (Not understanding why this will give rise to better substructure and reduced NP effects.) }

\subsection{Jet Algorithms for Deep-Inelastic Scattering}
\label{subsec: DIS-alg}
{\color{RoyalBlue}

Deep inelastic scattering (DIS) is defined by the process
\begin{equation}
\label{eq:DIS-Process}
    e^-(k) + P \rightarrow e^-(k') + X\,,
\end{equation}
where $P^\mu$ is the momentum of the incoming proton and $X$ denotes the hadronic final state. The standard DIS invariants are
\begin{equation}
\label{eq:DIS-invariants}
    Q^2 = -q^2\,, \qquad
    \xB = \frac{Q^2}{2 P\!\cdot q}\,, \qquad
    \ydis = \frac{P\!\cdot q}{P\!\cdot k}\,,
\end{equation}
with $q^\mu = k^\mu - k'^\mu$ the momentum of the exchanged virtual photon. Here $Q^2$ is the virtuality of the photon, $x$ is the Bjorken $x$ variable (fraction of momentum taken by the struck quark from the the incoming hadron), and $y$ is the energy transferred between leptonic and hadronic systems~\cite{Devenish:2004pb}.  {\color{blue}(Need to get some more information on invariants (particularly $y$.)} At leading order, the hadronic final state consists of a single outgoing parton,
\begin{equation}
\label{eq:DIS-LO}
    e^-(k) + p_{\mathrm{in}} \rightarrow e^-(k') + p_{\mathrm{out}}\,,
\end{equation}
where $p_{\mathrm{in}}^\mu$ and $p_{\mathrm{out}}^\mu$ are the momenta of the incoming and
outgoing partons, respectively.
The \emph{Breit frame} plays a central role in many DIS jet algorithms~\cite{Webber:1993bm}. It is defined such that the virtual photon carries no energy component,
\begin{equation}
\label{eq:DIS-photon}
    q^\mu = (0,0,0,-Q)\,,
\end{equation}
and is therefore purely space-like.  In this frame the proton momentum satisfies
$P^\mu = p_{\mathrm{in}}^\mu/x$, and the incoming and outgoing parton momenta take the simple form
\begin{equation}
\begin{aligned}
\label{eq:DIS-partons}
     p_{\mathrm{in}} ^\mu &= \frac{Q}{2}(1,0,0,+1)\,, \quad
    p_{\mathrm{out}}^{\mu} &= \frac{Q}{2}(1,0,0,-1)\,.
\end{aligned}
\end{equation}
Thus, in the Breit frame the proton and virtual photon collide head-on, as illustrated in Fig~\ref{fig:Breit-frame}. Within this frame, the proton remnant appears in the $\eta > 0$ region (the \emph{target hemisphere}), while the struck quark lies in the $\eta < 0$ region (the \emph{current hemisphere}). In practice, the jet clustering in this frame can lead to radiation from the proton remnant  being absorbed into the struck-quark jet, making the separation between the two hemispheres an important feature of DIS jet algorithms.
\begin{figure}[h]
    \centering
    \includegraphics[width=0.5\linewidth]{figures/Breit-frame.jpeg}
    \caption{Breit frame from Ref.~\cite{Arratia:2020ssx}}
    \label{fig:Breit-frame}
\end{figure}
The presence of an initial-state hadron requires a modification of the $e^+e^-$ jet algorithms described previously to account for the incoming beam. In sequential recombination algorithms, this can be achieved by introducing an additional quantity, the \emph{beam distance}, $d_{iB}$~\cite{Salam:2010nqg}
\footnote{In hadron–hadron collisions, two beam distances are defined, $d_{iB}$ and $d_{i\bar{B}}$, corresponding to the two incoming beams.}.} \\ \\
{\color{blue} (I think we need some information on the variables i.e. $x$, $y$, and $Q^2$, along with a few images of the lab frame DIS and the Breit frame DIS, and also some information on TFR and CFR.)}


\subsubsection{$k_\perp$ DIS Algorithm}
{\color{RoyalBlue}
As introduced in Ref.~\cite{Catani:1992zp}, the $k_\perp$ algorithm described in Section~\ref{subsec: e+e-} was extended to DIS. The procedure is defined in the Breit frame and is performed in two stages. The aim is to first cluster all particles into a \emph{beam jet} and  \emph{final-state macro-jets}, and then to resolve the jet sub-structure within the latter.
\begin{enumerate}
    \item \textbf{Pre-clustering into beam jet and macro-jets}
    \begin{enumerate}
        \item Define a hard-scattering scale, $E_t$, such that $E_t^2 \gg \Lambda_{\mathrm{QCD}}^2$.
        \item For each particle $i$, compute the beam distance:
        \begin{equation}
        \label{eq:DIS-durham-diB}
            d_{iB} = \frac{2E_i^2(1 - \cos\theta_{iB})}{E_t^2}\,,
        \end{equation}
        where $\theta_{iB}$ is the angle between particle $i$ and the beam (initial-state proton) direction.

        \item Compute the distance measure for every pair of particles $i$ and $j$:
        \begin{equation}
        \label{eq:DIS-durham-dij}
            d_{ij} = \frac{2(1 - \cos\theta_{ij})}{E_t^2}\min(E_i^2, E_j^2)\,.
        \end{equation}

        \item Identify the smallest value among $\{d_{ij}, d_{iB}\}$:
        \begin{itemize}
            \item If $d_{ij} < 1$ and is the smallest, combine $(p_i, p_j)$ into a pseudo-particle $p_{ij}$, using the $E$-scheme recombination scheme.
            \item If $d_{iB} < 1$ and is the smallest, assign $p_i$ to the beam jet.
        \end{itemize}
        \item Repeat the procedure iteratively from step~(b) for all particles and pseudo-particles not yet assigned to the beam jet, until all remaining objects satisfy $d_{ij}, d_{iB} > 1$. The result is a final set of clustered objects consisting of a beam jet and final-state macro-jets. {\color{red}(I am unsure what happens to the particles which are greater than 1. I think they are just macrojets. I am also unsure if there is 1 or multiple macrojets)}
    \end{enumerate}

    \item \textbf{Resolving the jet structure of the macro-jet}
    \begin{enumerate}
        \item Define a resolution parameter, $d_{\mathrm{cut}}$.
        \item For each particle within the final-state macro-jet, evaluate the distance measures $d_{ij}$ and apply the same clustering procedure as the $k_\perp$ algorithm for $e^+e^-$ annihilation, described in Section~\ref{subsec: e+e-} to the final-state macrojet.
    \end{enumerate}
\end{enumerate}
}


\subsubsection{Cambridge DIS Algorithm}
{\color{RoyalBlue}
The {\color{blue}(exclusive)} Cambridge algorithm for deep-inelastic scattering, originally proposed in Ref.~\cite{Wobisch:1998wt}, is an extension of the $e^+e^-$ Cambridge algorithm introduced in Section~\ref{subsec: e+e-}, generalised in the same manner as the $k_\perp$ algorithm was extended from $e^+e^-$ collisions to DIS, in the section above. \\\\
The procedure follows the same steps as the $k_\perp$ DIS algorithm described above, but uses angular \emph{ordering variables}, $v_{ij}$ and $v_{iB}$, in place of the distance measures $d_{ij}$ and $d_{iB}$. The variable $v_{ij}$ is defined in Eq.~\ref{eq:vij-cambridge}, while the beam angular-ordering variable $v_{iB}$ is given by:
\begin{equation}
    v_{iB} = 2(1 - \cos\theta_{iB})\,.
\end{equation}
The clustering proceeds by finding the smallest value among $\{v_{ij}, v_{iB}\}$ and determining whether the corresponding object is associated with a final-state macro-jet or the beam jet. Once the pre-clustering is complete, the algorithm proceeds as in the $e^+e^-$ Cambridge algorithm (see Section~\ref{subsec: e+e-}) to resolve the jet substructure within each macro-jet.
}
{\color{blue} (Add in: widely separated in angle are attached to the hard leg i.e. no multiple gluon clustering - soft jet freezing.)}







