\chapter{Resummation}
In this chapter we discuss the resummation of logarithmically enhanced contributions in perturbative QCD, focusing on the semi-numerical frameworks \textsc{CAESAR} (\emph{Computer Automated Expert Semi-Analytical Resummer})~\cite{Banfi:2004yd} and \textsc{ARES} (\emph{Automated Resummation for Event Shapes})~\cite{Banfi:2014sua}. These frameworks provide systematic methods to resum large logarithms that arise when fixed-order predictions become unreliable for a class of IRC safe observables known as event shapes. To illustrate the origin of such logarithms, we first examine a fixed-order calculation for the event shape known as thrust. We discuss what event shapes are, describing examples which we will use in Chapter~5. We then outline the \textsc{CAESAR} framework - firstly illustrated as an idea and then implemented into a code -  which automates resummation to next-to-leading logarithmic (NLL) accuracy, and the \textsc{ARES} framework, which extends this to next-to-next-to-leading logarithmic (NNLL) precision. \textsc{CAESAR} has been applied to both $e^+e^-$ and hadronic collisions~\cite{Banfi:2004yd, Banfi:2004nk}, while \textsc{ARES} has been developed primarily for $e^+e^-$ collisions~\cite{Banfi:2014sua, Banfi:2018mcq, Arpino:2019ozn}. In Chapter~5, we present the first application of \textsc{ARES} to hadronic collisions, including the treatment of initial-state radiation. \question{jet-veto for ARES also?}


\section{Motivation: Thrust and the Need for Resummation}
The event shape \emph{thrust}~\cite{Farhi:1977sg, Brandt:1964sa} is widely used in the literature and at the Large Electron-Positron Collider (LEP) for precise determinations of the strong coupling $\alpha_s$~\cite{Aglietti:2025jdj, Abbate:2010xh} \question{Unsure if these are the correct references to use due to the controversy?}. Thrust is defined as
\begin{equation}
\label{eq:thrust}
	T \equiv \max_{\vec{n}} \frac{\sum_j |\vec{p}_j \cdot \vec{n}|}{\sum_j |\vec{p}_j|}\,, \quad \tau \equiv 1 - T \,,
\end{equation}
where $\vec{p}_j$ denotes the three-momentum of particle $j$ in the final state, and the unit vector $\vec{n}$ that maximises this quantity is referred to as the \emph{thrust axis}. For $e^+e^- \to \mathrm{hadrons}$, thrust provides a measure of how ``pencil-like'' an event is: $T \simeq 1$ corresponds to two narrow, back-to-back jets, while smaller values of $T$ indicate more spherically distributed final states. It is conventional to work with the variable $\tau$, as defined in Eq.~\ref{eq:thrust}, such that the two-jet limit corresponds to the Born-level configuration with a perfectly back-to-back $q\bar{q}$ pair ($\tau \to 0$), while a spherically distributed multi-particle event occurs near $\tau \to 1/2$, as illustrated in Fig.~\ref{fig:thrust}.
\begin{figure}[h]
\begin{tikzpicture}[scale=0.9, >=Stealth]
\begin{scope}
\node at (0,2.5) {$\tau = 0$};

\draw[thick] (-3.5,0) -- (3.5,0);
\node[left]  at (-3.5,0) {$e^+$};
\node[right] at (3.5,0) {$e^-$};

\fill (0,0) circle (2pt);

\draw[very thick, ->] (0,0) -- (1.5,1.5) node[above right] {$q$};
\draw[very thick, ->] (0,0) -- (-1.5,-1.5) node[below left] {$\bar{q}$};

\node at (0,-2.7) {2-jet limit};
\end{scope}

\begin{scope}[xshift=8.5cm]
\node at (0,2.5) {$\tau \to \tfrac12$};

\draw[thick] (-3.5,0) -- (3.5,0);
\node[left]  at (-3.5,0) {$e^+$};
\node[right] at (3.5,0) {$e^-$};

\fill (0,0) circle (2pt);

\foreach \angle in {-150,-120,-90,-60,-30,0,30,60,90,120,150} {
  \draw[thick, ->] (0,0) -- ({2*cos(\angle)},{2*sin(\angle)});
  \draw[thick, ->] (0,0) -- ({-2*cos(\angle)},{2*sin(\angle)});
}

\node at (0,-2.7) {spherically distributed events};
\end{scope}
\end{tikzpicture}
\caption{Diagrammatic representation of thrust: $\tau \to 0$ corresponds to a pencil-like 2-jet configuration, while $\tau \to 1/2$ corresponds to a spherically distributed, multi-particle final state.}
\label{fig:thrust}
\end{figure}
\add{Maybe make the 2 jet limit arrows a bit shorter.}

\subsection*{Thrust: Fixed-order Calculation}
For the process $e^+e^- \to q\bar{q}$, the final state consists of two back-to-back partons, yielding the Born-level contribution
\begin{equation}
    \frac{1}{\sigma_0}\frac{d\sigma}{d\tau} = \delta(\tau),
\end{equation}
where $\sigma_0$ denotes the LO cross-section given in Eq.~\ref{eq:sigma-LO}. \\ \\
The first non-trivial contribution arises from the real-emission process $e^+e^- \to q\bar{q}g$. For a three-particle final state, thrust is determined by the largest momentum fraction,
\question{Should we be referencing the full cross-section example more? Also should we call it NLO real emission here? Different references say LO/NLO? Is there a difference with that and zeroth/1st order? Unsure what to call them basically.}
\begin{equation}
T = \max[x_q, x_{\bar{q}}, x_g],
\end{equation}
with $T_\text{min} = 2/3$ for equal momenta and $T_\text{max} = 1$. The corresponding differential distribution takes the form
\begin{equation}
	\frac{1}{\sigma_0}\frac{d\sigma}{dT} = C_F \frac{\alpha_s}{2\pi} \int dx_q \int dx_{\bar{q}} \frac{x_q^2 + x_{\bar{q}}^2}{(1-x_q)(1-x_{\bar{q}})} \, \delta\big(T - \max[x_q, x_{\bar{q}}, x_g]\big),
\end{equation}
\question{Same limits as in Sec.~\ref{sec:divergences} - also should we be referring to the equation in this section to give insight into where this equation came from?}
subject to the kinematic constraint $x_q + x_{\bar{q}} + x_g = 2$. Integrating over the available phase space yields
\begin{equation}
	\frac{1}{\sigma_0}\frac{d\sigma}{d\tau} = C_F \frac{\alpha_s}{2\pi} \left\{ \frac{3(1+\tau)(3\tau-1)}{\tau} + \frac{(4+6\tau(\tau-1))\ln\frac{1-2\tau}{\tau}}{\tau(\tau-1)} \right\}.
\end{equation}
As $\tau \to 0$, the distribution exhibits an IR divergence associated with soft and collinear emissions. This divergence is canceled by the virtual one-loop correction to $e^+e^- \to q\bar{q}$, in direct analogy with the cancellation discussed in Sec.~\ref{sec:divergences}, yielding a finite result as required by the IRC safety of thrust (Sec.~\ref{sec:IRC}). Using the regulated and integrated NLO cross-section given in Eq.~\ref{eq:sigma-full}, the complete distribution may be written as \question{maybe reference Schwartz here?}
\begin{equation}
\begin{aligned}
\label{eq:full-thrust-NLO}
    \frac{1}{\sigma_0}\frac{d\sigma}{d\tau} =\;& \delta(\tau) + C_F\frac{\alpha_s}{2\pi} \Bigg\{ \delta(\tau) \left( \frac{\pi^2}{3} - 1 \right) \\
    & + \left[ 3(1+\tau)(3\tau-1) + \frac{(4+6\tau(\tau-1))\ln(1-2\tau)}{1-\tau}\right] \left( \frac{1}{\tau} \right)_+  \\
    & - \frac{4+6\tau(\tau-1)}{1-\tau} \left( \frac{\ln \tau}{\tau} \right)_+ \Bigg\}.
\end{aligned}
\end{equation}
Neglecting the $\delta$-function contributions, the terms that become large in the $\tau \to 0$ limit may be referred to as \emph{logarithmically enhanced terms},
\begin{equation}
	\left(\frac{1}{\sigma_0}\frac{d\sigma}{d\tau}\right)_{\text{log-enh}} = C_F \frac{\alpha_s}{2\pi} \left[ -3 \left(\frac{1}{\tau}\right)_+ - 4 \left(\frac{\ln \tau}{\tau}\right)_+ \right].
\end{equation}
The corresponding cumulative distribution (cumulant) is given by
\begin{equation}
	\Sigma_{\text{log-enh}}(\tau) 
	= \frac{1}{\sigma_0} \int_0^\tau d\tau' \frac{d\sigma}{d\tau'} 
	= C_F \frac{\alpha_s}{2\pi} 
	\left( -2L^2 + 3L \right),
\end{equation}
where $L \equiv \ln(1/\tau)$. All remaining terms in Eq.~\eqref{eq:full-thrust-NLO} remain finite \add{in the cumulant} as $\tau \to 0$ and do not contribute to the logarithmic enhancement. \\ \\
These logarithmically-enhanced terms cause issues. Physically, the divergence of the cumulant as $\tau \to 0$ signals the breakdown of the fixed-order expansion, since an exactly two-jet configuration is unphysical, corresponding to the observation of bare quarks. \add{Plot of experimental vs. theory FO.} At higher orders in perturbation theory, the cumulant contains towers of logarithms of the form \question{Unsure if we can say towers or if it is a bit misleading for other areas of QCD?}
\begin{equation}
\begin{aligned}
\Sigma(\tau)\supset\quad&\alpha_sL^2 \qquad \alpha_s L \qquad \alpha_s \\
&\alpha_s^2L^4 \qquad \alpha_s^2 L^3 \qquad \alpha_s^2 L^2 \qquad \alpha_s^2 L \qquad \alpha_s^2 \\
&\alpha_s^3 L^6 \qquad \alpha_s^3 L^5 \qquad \dots \\
&\,\,\vdots \\
&\alpha_s^n L^{2n}\qquad\dots \qquad \qquad \qquad  \qquad  \qquad  \qquad \qquad  \qquad \qquad \alpha_s^n
\end{aligned}
\end{equation}
\add{Still a bit unsure by this format, sort of loose the $\alpha_s^n$?}
which spoil the convergence of the perturbative series in the small-$\tau$ region, where the bulk of the data lies. \\ \\
To restore predictive power, the perturbative expansion must therefore be reorganised and \emph{resummed}. Rather than expanding solely in powers of the strong coupling $\alpha_s$, one performs an expansion in powers of the logarithms $L = \ln(1/\tau)$. This reorganisation becomes essential in the region $\tau \ll 1$, where the logarithms are large and $\alpha_s L \sim 1$, causing fixed-order perturbation theory to break down. This procedure, known as \emph{resummation}, systematically accounts for these enhanced contributions to all orders in $\alpha_s$, yielding an exponentiated form for the cumulant,
\begin{equation}
	\Sigma(\tau) \simeq 
	\exp \Big\{ 
	L g_1(\alpha_s L) 
	+ g_2(\alpha_s L) 
	+ \alpha_s g_3(\alpha_s L) 
	+ \cdots 
	\Big\},
	\label{eq:resummed-cummulant}
\end{equation}
where the functions $g_i$ resum logarithms at successive logarithmic accuracies. The leading-logarithmic contributions $\alpha_s^n L^{n+1}$ are contained in $g_1(\alpha_s L) $, next-to-leading logarithms $\alpha_s^n L^{n}$ in $g_2(\alpha_s L) $, and so on. \\ \\
\reference{~\cite{Luisoni:2015xha, Schwartz:2014sze, ParticleDataGroup:2024cfk}}


\section{Event-shapes}
The appearance of large logarithms and the need for resummation are not unique to thrust. In fact, the resummed distribution shown in Eq.~\ref{eq:resummed-cummulant} applies to all IRC safe observables \question{rIRC, exclusive, global observables?}. Thrust is one such observable and belongs to a broader class known as \emph{event shapes}. Event shapes are IRC safe observables constructed from the four-momenta of final-state particles. They characterise the ``shape'' of an event by quantifying the geometrical distribution of its hadronic energy-momentum flow. While thrust provides a well-known example of an event shape in $e^+e^-$ collisions, event shapes are also widely used in hadronic collisions, including: \question{Correct description of event shape?} \\ \\
\question{The reference~\cite{Dasgupta:2003iq}, is a paper but was used merely for the background of event shapes - am unsure if necessary to cite or not?} 
\begin{itemize}
\item \textbf{Transverse thrust}, first introduced in Ref.~\cite{Banfi:2004nk}, is the extension of the $e^+e^-$ event-shape thrust, defined in Eq.~\eqref{eq:thrust}, to hadronic collisions. It is defined in the transverse plane as
\begin{equation}
  T_\perp \equiv \max_{\vec{n}_\perp}
  \frac{\sum_i \left| \vec{p}_{\perp i} \cdot \vec{n}_\perp \right|}
       {\sum_i \left| \vec{p}_{\perp i} \right|},
  \qquad
  \tau \equiv 1 - T_\perp \,,
\end{equation}
where $\vec{p}_{\perp i}$ denotes the transverse momentum of final-state particle $i$ with respect to the beam direction, and the unit vector $\vec{n}_\perp$ that maximises this quantity is referred to as the \emph{transverse thrust axis}.

\item \textbf{One-jettiness} is a special case of the more general $N$-jettiness event shape~\cite{Stewart:2010tn}, $\tau_N$, which quantifies how well the final state can be described by $N$ narrow jets. For $N=1$, one-jettiness is defined as
\begin{equation}
  \mathcal{T}_1 \equiv
  \sum_i
  \min\!\left[
    \frac{2 q_a \cdot p_i}{Q_a},
    \frac{2 q_b \cdot p_i}{Q_b},
    \frac{2 q_J \cdot p_i}{Q_J}
  \right] \,,
\end{equation}
where $q_{a,b,J}$ are the four-momenta of the incoming beams and the observed jet, and the sum runs over the final-state partons $p_i$. The normalisation factors are $Q_i = 2 \rho_i E_i$, with $\rho_i$ depending on the reference frame (e.g., $\rho_i = 1$ in the laboratory frame). While $\mathcal{T}_1$ is dimensionful, it is often convenient to define the dimensionless variable $\tau_1 \equiv \mathcal{T}_1 / Q$, where $Q$ is the hard scale of the interaction.

\item \textbf{Thrust minor}, also introduced in Ref.~\cite{Banfi:2004nk}, is a measure of out-of-plane radiation and is defined as
\begin{equation}
  T_m \equiv \frac{\sum_i |\vec p_{\perp i}\times \vec n_\perp|}{\sum_i p_{\perp i}}\,.
\end{equation}
\end{itemize}
\question{Correct description for all of them? \\ Are the references correct for transverse thrust and thrust minor?} \\ \\
Beyond their use in extracting parameters such as $\alpha_s$ and the QCD $\beta$-function, event shapes characterise the structure of final-state events. Event shapes provide detailed information on the distribution of energy and momentum, offering insight into features such as jet substructure. Because of their importance in probing QCD dynamics, accurate resummation techniques are essential. While many methods exist (see Sec.~\ref{subsec:QCD-ev}), much of the previous work on resumming event shapes has been observable-dependent. In this thesis, we focus on an observable-independent, \emph{semi-numerical} formalisms - \textsc{CAESAR} and \textsc{ARES}. We first review the \textsc{CAESAR} framework and then describe its extension to \textsc{ARES}. \\ \\
\reference{~\cite{Luisoni:2015xha, ParticleDataGroup:2024cfk, Banfi:2016yyq}}


\section{NNL Resummation: CAESAR}
Traditionally resummation up to NLL for final-state radiation was calculated for each observable manually. \textsc{CAESAR}, which was first outlined in ~\cite{} resums LL results for generic class of observables (requirements discussed below) where the resummation formula is expressed in terms of certain well-defined characteristics of the observable. The associated computer programme, presented in ~\cite{}, given a subroutine for an arbitrary observable observable, determines those characteristics enabling full automation of a large class of final state resummation in a range of processes. Take a born event consisting of $n$ hard partons where $n_i$ are the incoming legs. The observable $V(p_1,\dots p_n)$ are a function of the final-state momenta. Considering the resummation in the $n$-jet limit where essentially $n$-jet IRC safe observables measure the extent to which an event's energy flow departs from that of an $n$-parton event. For the resummation approaches to be valid, the observable $V$ should
\begin{enumerate}
	\item{vanish smoothly} as an extra $(n+1)^{\textrm{th}}$ with momentum $k$ is soft and or collinear to leg $\ell$. This can be parameterised as
	\begin{equation}
	V(\{\tilde{p}\},k)\simeq V_{\text{sc}}(k)\equiv d_\ell\left(\frac{k_t^{(\ell)}}{Q}\right)^ae^{-b_\ell\eta^{(\ell)}}g_\ell(\phi^{(\ell)})\,,
	\label{eq:general-sc-V}
\end{equation}
where $k_t^{(\ell)},\eta^{(\ell)},$ and $\phi^{(\ell)}$ are the transverse momentum, rapidity, and azimuthal angle of the emission $k$ with respect to the parent emitter $p_\ell$ and $a, b_{\ell}$ and $d_\ell$ are parameters. Note that $d_\ell$ is a coefficient which depends on the underlying Born kinematics, and we have introduced and arbitrary hard scale $Q$ which we use as a reference normalisation scale for transverse momenta. Please note that IRC safety guarantees that $a>0$ and $b_\ell>-a$.
\item{recursively IRC safe} which is essentially one step above IRC safety (see Sec.~\ref{sec:IRC}) and is that given an ensemble of arbitrary soft and collinear emissions, an additional emission which is relatively much softer and/or collinear should not significantly alter the value of the observable. Formally, the definition of rIRC safety can be seen as the following~\cite{}. Given the momenta $\kappa_i(\lambda_i)$, such that the observable is 
\begin{equation}
\label{eq:rIRC1}
	V(\{\tilde{p}\},\kappa_i(\lambda_i))=\lambda_i
\end{equation}
where in the soft and/or collinear limits, $\lambda_i\to0$, the azimuthal angle $\phi_i$ of the momenta $\kappa_i(\lambda_i)$ is fixed. The momentum functions $\kappa_i(\lambda_i)$ can all be different but must satisfy~\eqref{eq:rIRC1}. Now the conditions for rIRC safety are the two following
\begin{enumerate}
	\item The following limit must be well-defined and non-zero
	\begin{equation}
	\lim_{\varepsilon\to0}\frac {1}{\varepsilon}V(\{\tilde{p}\},\kappa_1(\varepsilon\lambda_1),\dots,\kappa_m(\varepsilon\lambda_m))\,,
	\end{equation}
	where this can be interpreted as the requirement that the soft and collinear scaling properties of the observable must be the same regardless of how many emissions there are.
	\item The following must hold
	\begin{equation}
	\begin{aligned}\lim_{\lambda_{m+1}\to0}\lim_{\varepsilon\to0}\frac {1}{\varepsilon}V(\{\tilde{p}\},\kappa_1(\varepsilon\lambda_1),\dots,\kappa_m(\varepsilon\lambda_m), \kappa_{m+1}(\varepsilon\lambda_{m+1}))\\
		= \lim_{\varepsilon\to0}\frac {1}{\varepsilon}V(\{\tilde{p}\},\kappa_1(\varepsilon\lambda_1),\dots,\kappa_m(\varepsilon\lambda_m))\,,
	\end{aligned}
	\end{equation}
	showing that the addition of an additional much softer and/or collinear emission must not affect the value of the observable.
\end{enumerate}
\item {Continuously global} where an observable is said to be global if it departs from zero for any emission of an $(n+1)^{\textrm{th}}$ parton that is not infinitely soft or collinear. Continuously global means that for a single soft emission, the observable's parametric dependence on the emission's transverse momentum should be the same everywhere in the phase-space, meaning that that the power $k_t$ should be the same everywhere.
\end{enumerate}
\add{Look at P.Monni Thesis for observable conditions. \\ Discuss event shape being 0 at $n$ narrow jets and refer to thrust.} \\ \\
Let us begin by looking at the $n$-jet cross section,
\begin{equation}
	\sigma_{\mathcal{H}}=\sum_{N=n-n_i}^\infty\int d\Phi_N\frac{d\sigma_N}{d\Phi_N}\mathcal{H}(q_1,\dots,q_N)\,,
\end{equation}
where $d\sigma_N/d\Phi_N$ is the differential cross section for producing $N$ final-state particles and $\mathcal{H}(q_1,\cdots,q_N)$ is \add{explain the cuts.}
The cumulant, $\Sigma_{\mathcal{H}}$, satisfying the cut as well as for events for which the observable is smaller than some value $v$
\begin{equation}
	\Sigma_{\mathcal{H}}(v)=\sum_N\int d\Phi_N\frac{d\sigma_N}{d\Phi_N}\Theta\left(v-V(q_1,\dots,q_N)\right)\mathcal{H}(q_1,\dots,q_N)
\end{equation}
The factorised form
\begin{equation}
\Sigma_{\mathcal{H}}\simeq\frac{1}{\sigma_{\mathcal{H}}}\int d\Phi_N\frac{d\sigma_N}{d\Phi_N}\Sigma\left(\{q_1,\dots,q_N\},v\right)\mathcal{H}(q_1,\dots,q_N)
\end{equation}
While we have introduced this for a general above \question{is it a above} $N$ jet event, in order to proceed more carefully it is convenient for us to look at a specific process namely that of $e^+e^-\to 3+\textrm{jet}$ events. We will discuss both \textsc{CAESAR} and \textsc{ARES} in the context of this, however please note that in Refs.~\cite{}, that these have been done for other processes. We wish to look at this process, since we use the resummation here as the premise of our final-state radiation for the resummation of hadronic collision for $H+\textrm{jet}$ events, as discussed in Chapter.~5 hence it is convenient to discuss it here. \\ 
At NLL accuracy for three-jet observables, we find that the cumulant is
\begin{equation}
\Sigma\left(\{p_1,p_2,p_3\},v\right)=e^{-R_{\mathrm{NLL}}(v)}\mathcal{F}_{\mathrm{NLL}}\left(R'_{\mathrm{NLL}}(v)\right)\,,\quad R'_{\mathrm{NLL}}\simeq-v\frac{dR_{\mathrm{NLL}}}{dv}\,,
\end{equation}
where $R_{\mathrm{NLL}}$ is the NLL radiator. This encodes the probability of observing no emissions $k_i$ with $V(\{\tilde{p}\}, k_i)>v$. The expression is a sum of contributions from the three dipoles which build up a three-jet configuration $q\bar{q}, qg, g\bar{q}$:
\begin{equation}
\label{eq:NLL-radiator}
\begin{split}
	R_{\mathrm{NLL}}(v)=\sum_{(ij)\in\{(q\bar{q}), (qg), (g\bar{q})\}}C_{(ij)}\left(\sum_{\ell=i,j}\left[r_\ell(L)+r'_\ell(L)\left(\langle \ln(d_\ell g_\ell)\rangle - b_\ell\ln\frac{2E_\ell}{Q}\right) \right. \right. & \\
	\left. \left. +B_\ell T\left(\frac{L}{a+b_\ell}\right)\right] +2\ln\frac{Q_{ij}}{Q}T\left(\frac{L}{a}\right)\right)\,,
\end{split}
\end{equation}
where $L\equiv\ln(1/v)$ and $C_{(ij)}$ is the colour factor associated with the dipole. 





\todo{Add in the exact N jet cumulant and explain all a bit more throughly. Then talk about how we want to look at 3-jet specifically as it gives the foundation for the FSR for Chapter 5. Then discuss CAESAR for this process and then ARES for this procedure also. The radiator and hard collinear counter term may need to change a bit as we have changed them slightly in our paper - but we can just clarify this a bit. I think it is better to keep the CAESAR and FSR ARES in this chapter regardless as that is not original research - at the most its a reformulation. The next chapter should include the ISR (maybe - need to discuss with A Banfi) and the observables that we calculated and the phenomenology. \\ \\ Also add cumulant in front of $|\mathcal{M}|^2$ and then go to give exact expression for $\mathcal{F}_{\mathrm{NLL}}$ and the radiator. Also give the kinematics.}



\begin{description}
   \item[Continuously global] A global observable is one which is sensitive to all of the emissions in an event. An observable which is continuously global adds another feature such that the scaling of the observable with respect to the energy of the emission is the same everywhere in the phase space. If an observable is not continuously global, it results in the presence of additional large logarithms known as non-global logarithms (these will not be discussed in this report).
 \item[rIRC safety] In section \ref{sec:IRC}, we explained the concept of IRC safety. rIRC safety further constrains the observable, such that it is unaffected by additional soft and/or collinear emissions at widely separated scales. IRC safety involves two energy scales: the hard partons and the additional soft-collinear emissions. rIRC introduces a third (softer) energy scale, requiring that the observable remains insensitive to emissions in this new region of the phase space. This allows us to split the emissions into a set that are resolved and a set that are unresolved. 
 
\end{description}



\section{NNL Resummation: ARES}
\todo{NNLL: Formalism + ISR prescription \\ At the very least here we need to discuss the hard collinear thing which helps us with ISR for our project.}

\subsection*{Formulation}
The ARES method begins by understanding that any rIRC safe observable $V$, in the presence of a single soft emission collinear to leg $\ell$, can be parameterised in the following way \cite{Arpino:2019ozn}:
\begin{equation}
	V(\{p\},k)\simeq V_{\text{sc}}(k)\equiv d_\ell\left(\frac{k_t^{(\ell)}}{Q}\right)^ae^{-b_\ell\eta^{(\ell)}}g_\ell(\phi^{(\ell)})\,,
	\label{eq:general-sc-V}
\end{equation}
where $k_t^{(\ell)},\eta^{(\ell)},$ and $\phi^{(\ell)}$ are the transverse momentum, rapidity, and azimuthal angle of the emission $k$ with respect to the parent emitter $p_\ell$ and $a, b_{\ell}$ and $d_\ell$ are parameters. Note that $d_\ell$ is a coefficient which depends on the underlying Born kinematics, and we have introduced and arbitrary \enquote{resummation} scale $Q$ which we use as a reference normalisation scale for transverse momenta. 
 $a, b_{\ell}, d_\ell, g_\ell$ 
%Unsure if its best to put this here. 
\\ \\
Furthermore, we must consider an observable $V(\{\tilde{p}\},k_1,\cdots , k_n)$ which is a non-negative function of final-state $\{\tilde{p}\}$ and additional emissions $\{k_i\}$. Any continuously global, rIRC safe observable $V$ goes smoothly to zero for the born event such that:
\begin{equation}
V(\tilde{p}_1,\cdots , \tilde{p}_n)=V(p_1,\cdots ,p_n)=0\, .
	\label{eq:Born-observable}
\end{equation}
It is clear to see that the tilde denotes the final-state moment recoil due to the extra emissions and that there is a mapping between the recoiled and the Born momenta.
Additionally, in our case of the three-jet observable, we expect the observable to approach zero for momentum configurations that approach the limit of 3 narrow jets. 
We consider the cumulative distribution of a three-jet event shape $V(p_1,\cdots ,p_n)$, defined as the following:
\begin{equation}
	\Sigma_{\mathcal{H}}(v)\equiv\frac{1}{\sigma_{\mathcal{H}}}\sum_{n=3}^{\infty}\int d\Phi_n\frac{d\sigma_n}{d\Phi_n}\mathcal{H}(p_1,\cdots ,p_n)\,\Theta(v-V(p_1,\cdots ,p_n))\,,
	\label{eq:hard-cumulant}
\end{equation}
where $d\Phi_n$ is the $n$-particle phase space, $d\sigma_n/d\Phi_n$ is the normalised differential cross-section for producing $n$ final-state particles, and 
$\mathcal{H}(p_1,\cdots ,p_n)$ is an event selection function which is 1 for events with 3 or more hard jets and 0 otherwise. This can be calculated through jet algorithms but we will not be discussed in this report. \\ \\
In near-to-planar kinematics i.e. when $v\ll1$, we find that $\Sigma_{\mathcal{H}}(v)$ becomes the following:
 \begin{equation}
	\Sigma_{\mathcal{H}}(v)\simeq\frac{1}{\sigma_{\mathcal{H}}}\int d\Phi_3\frac{d\sigma_3}{d\Phi_3}\Sigma(\{p_1,p_2,p_3\},v)\,\mathcal{H}(p_1,p_2,p_3)\,.
	\label{eq:hard-cumulant-approx}
\end{equation}
In the region where $v\ll1$, large logarithms emerge in the function $\Sigma(\{\tilde{p}_1,\tilde{p}_2,\tilde{p}_3\},v)$. Our aim is to be able to resum these logarithms, up to a given logarithmic accuracy to all orders in the strong coupling (in our case NNLL). In fact, we find that this is equal to the following:
\begin{equation}
	\Sigma(\{p_1,p_2,p_3\},v)\equiv\mathcal{V}(\{p\})\sum_{n=0}^\infty S(n)\int\prod_{i=1}^n[dk_i]\,|\mathcal{M}(\{\tilde{p}\},k_1,\cdots ,k_n)|^2\,\Theta(v-V(\{\tilde{p}\},k_1,\cdots ,k_n)\, ,
	\label{eq:born-cumulant}
\end{equation}
where $S(n)$ is the symmetry factor e.g. $1/n!$ for $n$ identical gluons. Moreover, $\mathcal{V}(\{\tilde{p}\}$ includes all virtual corrections to the Born process, and $|\mathcal{M}(\{\tilde{p}\},k_1,\cdots ,k_n)|^2$ contains all real corrections.
Using properties of rIRC safety and derivations found in \cite{Banfi:2018mcq}, we find that the NNLL master formula, in the presence of initial soft radiation only, is the following:
\begin{multline}
  \label{eq:Sigma-NNLL-fs}
  \Sigma(\{p_1,p_2,p_3\},v)= e^{-R_{\mathrm{NNLL}}(v)}\left(\mathcal{F}_{\mathrm{NLL}}(\lambda)(1+\frac{\alpha_s(Q)}{2\pi}H^{(1)}
      +\frac{\alpha_s(Qv^{\frac{1}{a+b_3}})}{2\pi}C_{\mathrm{hc},3}^{(1)}) +\frac{\alpha_s(Q)}{\pi} \delta\mathcal{F}_{\mathrm{NNLL}}(\lambda)\right)
\end{multline}
where $\lambda\equiv \alpha_s(Q)\beta_0\ln(1/v)$ and
$\beta_0=(11 C_A\!-2 n_f)/(12\pi)$. 
In this formula, \(\mathcal{F}_{\mathrm{NLL}}\) and \(\delta\mathcal{F}_{\mathrm{NNLL}}\) have a general expression for any rIRC safe observable and are single logarithmic functions. This allows us to express them in terms of the single-logarithmic quantity \(\lambda\). We find that \(\mathcal{F}_{\mathrm NLL}\) is purely NLL, while the terms containing \(H^{(1)}, C_{\mathrm{hc},3}^{(1)}\), and \(\delta\mathcal{F}_{\mathrm NNLL}\) are of order NNLL. \\ \\
The idea is that at NLL, there are a collection of soft-collinear emissions. 
At NNLL, we still have this ensemble of soft-collinear emission contributing terms of $(\alpha_s^n L^n)$. However, in addition to this, there will be \emph{one} of the following: hard-collinear emission, soft wide-angle emission or, gluon splitting emission. Each one will contribute a factor of the coupling such that the terms are of order $\alpha_s^{n+1} L^n$, achieving NNLL accuracy.
We will now describe all the quantities within this master formula:
\begin{itemize}
\item the ``radiator'' $R_{\mathrm{NNLL}}(v)$ includes all virtual
  corrections of soft and/or collinear origin. These are divergent,
  but their divergences cancel with those corrections from real radiation. We will discuss this in a little more detail later in section \ref{sec:radiator}.
\item \emph{Resolved} soft and collinear emissions widely separated in
  angle, and the corresponding virtual corrections, build up to the
  NLL function $\mathcal{F}_{\mathrm{NLL}}(\lambda)$. To compute this
  function one must obtain the expression for the observable
  $V(\{\tilde p\},k_1,\dots,k_n)$ when $k_1,\dots, k_n$ are soft
  \emph{and} collinear. In particular, for rIRC safe observables, the
  following limit exists
  \begin{equation}
    \label{eq:V-lim}
    \lim_{v\to 0}\frac{V(\{\tilde p\},k_1,\dots,k_n)}{v} \equiv \frac{V^{\mathrm sc}(\{\tilde p\},k_1,\dots,k_n)}{v}\,.
  \end{equation}
  Given $V^{\mathrm{sc}}(\{\tilde p\},k_1,\dots,k_n)$, the function $\mathcal{F}_{\mathrm NLL}(\lambda)$ can be computed using the general formula
 \begin{equation}
    \label{eq:FNLL-formula}
    \mathcal{F}_{\mathrm{NLL}}(\lambda) = \int\! \dZ \,\Theta\left(1-\frac{\Vsc{\{k_i\}}}{v}\right)\,,
  \end{equation}
  where we have defined the soft-collinear measure $\dZ$ through the formula
  \begin{multline}
    \label{eq:dZ-def}
    \int\! \dZ \,G(\{k_i\}) \\ \equiv \lim_{\epsilon\to 0} \epsilon^{\RpNLL} \sum_{n=0}^\infty\frac{1}{n!} \int \prod_{i=1}^n \left(\sum_{\ell_i} R'_{\text{NLL},\ell_i}\int_{\epsilon}^\infty \frac{d\zeta_i}{\zeta_i}\int_0^{2\pi}\frac{d\phi_i^{(\ell_i)}}{2\pi}\right)G(k_1,\dots,k_n)\,.
\end{multline}
In the above formula, we have introduced the rescaling variables
$\zeta_i\equiv V^{\mathrm{sc}}(\{\tilde p\},k_i)/v$. With
$\{\zeta_i,\phi_i^{(\ell_i)}\}$ fixed for event shapes, one can freely
integrate over emissions' rapidities, which yields a jacobian factor
$R'_{\text{NLL},\ell_i}$ for each emission and for each emitting leg.

\item $H^{(1)}$ and $C_{\mathrm{hc},3}^{(1)}$ are coefficients of order $\alpha_s(Q)$ due to the virtual corrections. Multiplying these by $\mathcal{F}_{\mathrm{NLL}}(\lambda)$, we obtain terms of order NNLL. 

\item All other \emph{resolved} soft or final-state hard-collinear emissions and the corresponding virtual corrections build the NNLL function
  $\delta\mathcal{F}_{\mathrm NNLL}(\lambda)$. More precisely, these
  consist of a single soft and/or collinear ``special'' emission,
  accompanied by an arbitrary number of soft and collinear emissions
  widely separated in angle (and the corresponding virtual
  corrections). The function $\delta\mathcal{F}_{\mathrm NNLL}(\lambda)$
  is the sum of various terms, and comes from relaxing different approximations that were used within the NLL calculation. %see what this means by the NLL calculation and reword this part.
  For a three-jet event shape we find this to be the following:
\begin{equation}
    \label{eq:dFNNLL}
    \delta\mathcal{F}_{\mathrm NNLL} = \delta\mathcal{F}_{\mathrm sc}+\delta\mathcal{F}_{\mathrm hc}+\delta\mathcal{F}_{\mathrm rec}+\Delta\mathcal{F}_{\mathrm rec}+\delta\mathcal{F}_{\mathrm wa}+\Delta\mathcal{F}_{\mathrm wa}+\delta\mathcal{F}_{\mathrm correl}\,,
\end{equation}
where each contribution reflects the region of phase space spanned by the special emission.%Work out what this special emission is. 

Note that the three-jet event introduces two new functions $\Delta\mathcal{F}_{\mathrm{wa}}$ and $\Delta\mathcal{F}_{\mathrm{rec}}$ which do not occur in the two-jet scenario. All these contributions within~\ref{eq:dFNNLL} can be expressed in terms of integrals over the soft-colliner measure $\dZ$ of the appropriate weight function $G(\{k_i\})$, as shown in equation~(\ref{eq:dZ-def}). 
\end{itemize}


\subsection*{The Radiator}
\label{sec: radiator}
At NNLL accuracy, with more than two emitting legs, it is convenient
to split the radiator $R_{\NNLL}(v)$ into the sum of a soft part $R_s(v)$ and a hard-collinear part $R_{\mathrm hc}(v)$:
\begin{equation}
  \label{eq:nnll-radiator}
  R_{\NNLL}(v) = R_s(v)+R_{\mathrm hc}(v)\,.
\end{equation}
The soft radiator $R_s$ includes contributions due to the unresolved soft and/or collinear radiation such that the radiator includes terms of order LL and above. The hard-collinear radiator, on the other hand, is due to the unresolved hard-collinear radiation and therefore contributes terms of NLL accuracy and above. In fact it is found in \cite{Arpino:2019ozn}, a master formula for  $R_{\mathrm NNLL}(v)$, whose explicit expression depends only on the observable properties with a single soft and collinear emission $a,b_\ell,d_\ell,g_\ell(\phi)$, which are found using equation~\ref{eq:general-sc-V}. The soft radiator is found to be:
\begin{equation}
R_s(v)=\sum_{(ij)}C_{(ij)}\sum_{\ell\in(ij)}\mathcal{R}_{\ell}^{(ij)}(v)\,,
\label{eq:soft-radiator}
\end{equation}
where $C_{(ij)}$ is the Casimir operator 
such that in our case, $C_{(q\bar{q})}=2C_F-C_A$ and $C_{(gq)}=C_{(g\bar{q})}=C_A$. The full expression for $\mathcal{R}_{\ell}^{(ij)}$ i.e. the soft radiator for each dipole is, is contained in \cite{Arpino:2019ozn}.
The hard-collinear radiator is found to be the following:
\begin{equation}
	R_{\mathrm{hc}}(v)=\sum_{\ell}R_{\mathrm{hc},\ell}(v)\,,
	\label{eq:radiator-hc}
\end{equation}
such that
\begin{equation}
	R_{\mathrm{hc},\ell}(v)=-h_2^{(\ell)}(\lambda)-\frac{\alpha_s}{\pi}h_3^{(\ell)}(\lambda)\,,
	\label{eq:radiator-hc-more}
\end{equation}
where the functions $h_2^{(\ell)}, h_3^{(\ell)}$ depend solely on $a$ and $b_{\ell}$ and are again contained within \cite{Arpino:2019ozn}.

\subsection*{$\delta\mathcal{F}_{\mathrm{NNLL}}$}
%Re-word this title
The corrections, $\delta\mathcal{F}_{\mathrm{sc}}, \delta\mathcal{F}_{\mathrm{wa}}, \Delta\mathcal{F}_{\mathrm{wa}}, \delta\mathcal{F}_{\mathrm{correl}}$ all have origin in resolved soft emissions, whereas the terms $\delta\mathcal{F}_{\mathrm{hc}}, \delta\mathcal{F}_{\mathrm{rec}}, \Delta\mathcal{F}_{\mathrm{rec}}$ have hard-collinear nature. 
% Each resolved real emission is found by requiring that $V(\{\tilde{p}\},k_i)>\epsilon v$ such that there is a lower bound on the phase-space where a resolved emission can take place and because of rIRC safety, $\epsilon$ is independent of $v$. 

\begin{enumerate}
\item[Soft emissions]: $\delta\mathcal{F}_{\mathrm sc}$ represents the running coupling corrections to the real emissions in the CMW scheme. It also accounts for the correct rapidity boundary for a single soft-collinear emission. $\delta\mathcal{F}_{\mathrm wa}$ arises from a soft emission near $\eta=0$ and accounts for the variation between the observable and its soft-collinear parametrisation for a single soft wide-angle emission accompanied by an ensemble of soft-collinear gluons. $\Delta\mathcal{F}_{\mathrm wa}$ is the NNLL contribution arising from wide angled resolved emissions. Finally, $\delta\mathcal{F}_{\mathrm correl}$ addresses the treatment of the splitting of a single soft and collinear gluon.
\item[Hard-collinear emissions]: $\delta\mathcal{F}_{\mathrm hc}$ originates from the emission of a hard-collinear parton at the level of the squared matrix element. $\delta\mathcal{F}_{\mathrm rec}$ is due to a hard-collinear emission (at the level of the kinematics) recoiling against the soft-collinear ensemble of emissions. This includes contributions due to spin correlations and hence may depend on the underlying Born momenta. $\Delta\mathcal{F}_{\mathrm rec}$ appears for the first time in near-to-planar three jet events. This accounts for correlations which occur between the spin of a hard collinear gluon and the event plane. 
\end{enumerate}
\subsection*{Initial State Radiation}
 {\color{purple} For the very first time NNLL resummation has been performed for Hadronic collisions using the ARES method.}

\section{Validation and Matching}
\emph{Include something on matching with fixed order i.e. something to essentially show that resummation is necessary. I think here we can add in some of the plots calculated from the numerical work done in hhHjet project.}